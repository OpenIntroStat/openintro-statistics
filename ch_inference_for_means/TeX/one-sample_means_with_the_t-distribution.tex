\exercisesheader{}

% 1

\eoce{\qt{Identify the critical $t$\label{identify_critical_t}} An independent random 
sample is selected from an approximately normal population with unknown 
standard deviation. Find the degrees of freedom and the critical $t$-value 
(t$^\star$) for the given sample size and confidence level.
%\begin{multicols}{4}
\begin{parts}
\item $n = 6$, CL = 90\%
\item $n = 21$, CL = 98\%
\item $n = 29$, CL = 95\%
\item $n = 12$, CL = 99\%
\end{parts}
%\end{multicols}
}{}

% 2

\eoce{\qt{$t$-distribution\label{t_distribution}}
The figure on the right shows three 
unimodal and symmetric curves:
the standard normal (z) distribution,
the $t$-distribution with 5 degrees of freedom,
and the $t$-distribution with 1 degree of freedom.
Determine which is which, and explain your reasoning.
\begin{center}

\FigureFullPath[Three distributions are shown, all symmetric, bell-shaped, and centered at zero. The first is shown as a solid line and has the broadest peak of the three distributions, and the tails of this distribution also visually approach zero at about -3 and positive 3. The second curve that is shown as a dashed line has a less broad, slightly sharper peak than the distribution based on solid line. The tails of the distribution with the dashed line has tails that visually approach zero at values of about -4 and positive 4. The third curve is shown as a dotted line and has the sharpest peak of the three distributions. The tails of the dotted line distribution has tails that visually approach zero further out, beyond the limits shown in this plot of -4 and positive 4.]{0.4}{ch_inference_for_means/figures/eoce/t_distribution/t_distribution}
\end{center}
}{}

% 3

\eoce{\qt{Find the p-value, Part I\label{find_T_pval_1_2_sided}}
An independent random sample 
is selected from an approximately normal population
with an unknown standard 
deviation.
Find the p-value for the given sample size and test statistic.
Also determine if the null hypothesis would be rejected at 
$\alpha = 0.05$.
\begin{parts}
\item $n = 11$, $T = 1.91$
\item $n = 17$, $T = -3.45$
\item $n = 7$, $T = 0.83$
\item $n = 28$, $T = 2.13$
\end{parts}
}{}

% 4

\eoce{\qt{Find the p-value, Part II\label{find_T_pval_2_2_sided}}
An independent random sample 
is selected from an approximately normal population
with an unknown standard 
deviation.
Find the p-value for the given sample size and test statistic.
Also determine if the null hypothesis would be rejected at 
$\alpha = 0.01$.
\begin{parts}
\item $n = 26$, $T = 2.485$
\item $n = 18$, $T = 0.5$
\end{parts}
}{}

% 5

\eoce{\qt{Working backwards, Part I\label{work_backwards_1}} A 95\% confidence 
interval for a population mean, $\mu$, is given as (18.985, 21.015). This 
confidence interval is based on a simple random sample of 36 observations. 
Calculate the sample mean and standard deviation. Assume that all conditions 
necessary for inference are satisfied. Use the $t$-distribution in any 
calculations.
}{}

% 6

\eoce{\qt{Working backwards, Part II\label{work_backwards_2}} A 90\% confidence 
interval for a population mean is (65, 77). The population distribution is 
approximately normal and the population standard deviation is unknown. This 
confidence interval is based on a simple random sample of 25 observations. 
Calculate the sample mean, the margin of error, and the sample standard 
deviation.
}{}

\D{\newpage}

% 7

\eoce{\qt{Sleep habits of New Yorkers\label{ny_sleep_habits_2_sided}}
New York is known as 
``the city that never sleeps".
A random sample of 25 New Yorkers were asked how 
much sleep they get per night.
Statistical summaries of these data are shown 
below.
The point estimate suggests New Yorkers sleep less than
8~hours a night on average.
Is the result statistically significant?
\begin{center}
\begin{tabular}{rrrrrr}
 \hline
n   & $\bar{x}$ & s     & min   & max \\ 
 \hline
25  & 7.73      & 0.77  & 6.17  & 9.78 \\ 
  \hline
\end{tabular}
\end{center}

\begin{parts}
\item Write the hypotheses in symbols and in words.
\item Check conditions, then calculate the test statistic, $T$, and the 
associated degrees of freedom.
\item Find and interpret the p-value in this context. Drawing a picture may be 
helpful.
\item What is the conclusion of the hypothesis test?
\item If you were to construct a 90\% confidence interval that corresponded to 
this hypothesis test, would you expect 8 hours to be in the interval?
\end{parts}
}{}

% 8

\eoce{\qt{Heights of adults\label{adult_heights}}
Researchers studying anthropometry 
collected body girth measurements and skeletal diameter measurements, as well as 
age, weight, height and gender, for 507 physically active individuals. The 
histogram below shows the sample distribution of heights in centimeters. 
\footfullcite{Heinz:2003} \\
\begin{minipage}[c]{0.75\textwidth}
\begin{center}
\FigureFullPath[A histogram is shown for "Height" with values ranging from  140 to 200, with a bin width of 5. The distribution is roughly symmetric with a center at about 170. The bin heights, starting with the bin from 145 to 150, are about 3, 17, 55, 70, 100, 85, 95, 50, 30, 15, and 3.]{}{ch_inference_for_means/figures/eoce/adult_heights/adult_heights_hist}
\end{center}
\end{minipage}
\begin{minipage}[c]{0.23\textwidth}
\begin{center}
\begin{tabular}{l|r l}
Min     & 147.2 \\
Q1      & 163.8 \\
Median  & 170.3 \\
Mean    & 171.1 \\
SD      &  9.4 \\
Q3      & 177.8 \\
Max     & 198.1 \\
\end{tabular}
\end{center}
\end{minipage}
\begin{parts}
\item What is the point estimate for the average height of active individuals? 
What about the median?
\item What is the point estimate for the standard deviation of the heights of 
active individuals? What about the IQR?
\item Is a person who is 1m 80cm (180 cm) tall considered unusually tall? And is 
a person who is 1m 55cm (155cm) considered unusually short? Explain your 
reasoning.
\item The researchers take another random sample of physically active 
individuals. Would you expect the mean and the standard deviation of this new 
sample to be the ones given above? Explain your reasoning.
\item The sample means obtained are point estimates for the mean height of all 
active individuals, if the sample of individuals is equivalent to a simple 
random sample.
What measure do we use to quantify the variability of such an estimate?
Compute 
this quantity using the data from the original sample under the condition that 
the data are a simple random sample. 
\end{parts}
}{}

% 9

\eoce{\qt{Find the mean\label{find_mean_2_sided}}
You are given the following hypotheses:
\begin{align*}
H_0&: \mu = 60 \\
H_A&: \mu \neq 60
\end{align*}
We know that the sample standard deviation is 8
and the sample size is 20.
For what sample mean would the p-value be equal to 0.05?
Assume that all conditions necessary for inference are satisfied.
}{}

\D{\newpage}

% 10

\eoce{\qt{$t^\star$ vs. $z^\star$\label{critical_t_vs_z}} For a given confidence 
level, $t^{\star}_{df}$ is larger than $z^{\star}$. Explain how $t^{*}_{df}$ 
being slightly larger than $z^{*}$ affects the width of the confidence interval.
}{}

% 11

\eoce{\qt{Play the piano\label{play_piano_2_sided}}
Georgianna claims that in a small city 
renowned for its music school, the average child takes less than 5 years of 
piano lessons. We have a random sample of 20 children from the city, with a 
mean of 4.6 years of piano lessons and a standard deviation of 2.2 years.
\begin{parts}
\item
    Evaluate Georgianna's claim (or that the opposite might be true)
    using a hypothesis test.
\item
    Construct a 95\% confidence interval for the number of years
    students in this city take piano lessons, and interpret it
    in context of the data.
\item
    Do your results from the hypothesis test and the confidence
    interval agree?
    Explain your reasoning.
\end{parts}
}{}

% 12

\eoce{\qt{Auto exhaust and
    lead exposure\label{auto_exhaust_lead_exposure_2_sided}} 
Researchers interested in lead exposure due to car exhaust
sampled the blood of 52 police officers subjected to constant
inhalation of automobile exhaust fumes while working traffic
enforcement in a primarily urban environment.
The blood samples of these officers had an average lead
concentration of 124.32 $\mu$g/l and a SD of 37.74 $\mu$g/l;
a previous study of individuals from a nearby suburb,
with no history of exposure, found an average blood level
concentration 
of 35 $\mu$g/l.\footfullcite{Mortada:2000}
\begin{parts}
\item
    Write down the hypotheses that would be appropriate for
    testing if the police officers appear to have been exposed
    to a different concentration of lead.
\item\label{auto_exhaust_lead_exposure_2_sided_cond}
    Explicitly state and check all conditions necessary for
    inference on these data.
\item
    Regardless of your answers in
    part~(\ref{auto_exhaust_lead_exposure_2_sided_cond}),
    test the hypothesis that the downtown police officers have
    a higher lead exposure than the group in the previous study.
    Interpret your results in context.
\end{parts}
}{}

% 13

\eoce{\qt{Car insurance savings\label{car_insurance_savings}}
A market researcher wants to evaluate car insurance savings
at a competing company.
Based on past studies he is assuming that the standard
deviation of savings is \$100.
He wants to collect data such that he can get a margin of
error of no more than \$10 at a 95\% confidence level.
How large of a sample should he collect?
}{}

% 14

\eoce{\qt{SAT scores\label{sat_scores_CI}}
The standard deviation of SAT scores for students at
a particular Ivy League college is 250 points.
Two statistics students, Raina and Luke, want to estimate
the average SAT score of students at this college as part
of a class project.
They want their margin of error to be no more than 25 points.
\begin{parts}
\item
    Raina wants to use a 90\% confidence interval.
    How large a sample should she collect?
\item
    Luke wants to use a 99\% confidence interval.
    Without calculating the actual sample size, determine
    whether his sample should be larger or smaller
    than Raina's, and explain your reasoning.
\item
    Calculate the minimum required sample size for Luke.
\end{parts}
}{}
