


%_______________
\newpage\subsection*{Exercises} % Comparing many means with ANOVA

% 1

\eoce{\qt{Fill in the blank\label{fitb_anova}} When doing an ANOVA, you observe 
large differences in means between groups. Within the ANOVA framework, this 
would most likely be interpreted as evidence strongly favoring the \underline{\hspace{20mm}} hypothesis.
}{}

% 2

\eoce{\qtq{Which test\label{which_test_anova}} We would like to test if 
students who are in the social sciences, natural sciences, arts and 
humanities, and other fields spend the same amount of time studying for 
this course. What type of test should we use? Explain your reasoning.
}{}

% 3

\eoce{\qt{Chicken diet and weight, Part III\label{chick_wts_anova}} In Exercises~\ref{chick_wts_linseed_horsebean} and \ref{chick_wts_casein_soybean} we compared the effects of two types of feed at a time. A better analysis would first consider all feed types at once: casein, horsebean, linseed, meat meal, soybean, and sunflower. The ANOVA output below can be used to test for differences between the average weights of chicks on different diets.
\begin{center}
\begin{tabular}{lrrrrr}
\hline
        & Df    & Sum Sq        & Mean Sq   & F value   & Pr($>$F) \\ 
\hline
feed        & 5     & 231,129.16    & 46,225.83     & 15.36     & 0.0000 \\ 
Residuals   & 65 & 195,556.02   & 3,008.55  &       &  \\ 
\hline
%\multicolumn{6}{r}{$s_{pooled} = 55.85$ on $df=65$}
\end{tabular}
\end{center}
Conduct a hypothesis test to determine if these data provide convincing evidence that the average weight of chicks varies across some (or all) groups. Make sure to check relevant conditions. Figures and summary statistics are shown below.

\begin{minipage}[c]{0.65\textwidth}
\begin{center}
\includegraphics[width= \textwidth]{ch_inference_for_means/figures/eoce/chick_wts_anova/chick_wts_box.pdf}
\end{center}
\end{minipage}
\begin{minipage}[c]{0.35\textwidth}
{\footnotesize\begin{tabular}{l c c c}
\hline
            & Mean      & SD        & n \\
\hline
casein          & 323.58        & 64.43 & 12 \\
horsebean   & 160.20        & 38.63 & 10 \\
linseed         & 218.75        & 52.24 & 12 \\
meatmeal    & 276.91        & 64.90 & 11 \\
soybean         & 246.43        & 54.13 & 14 \\
sunflower       & 328.92        & 48.84 & 12 \\
\hline
\end{tabular}}
\end{minipage}
}{}

% 4

\eoce{\qt{Teaching descriptive statistics\label{teach_descriptive_stats}} A study 
compared five different methods for teaching descriptive statistics. The five 
methods were traditional lecture and discussion, programmed textbook 
instruction, programmed text with lectures, computer instruction, and computer 
instruction with lectures. 45 students were randomly assigned, 9 to each 
method. After completing the course, students took a 1-hour exam. 
\begin{parts}
\item What are the hypotheses for evaluating if the average test scores are 
different for the different teaching methods?
\item What are the degrees of freedom associated with the $F$-test for 
evaluating these hypotheses?
\item Suppose the p-value for this test is 0.0168. What is the conclusion?
\end{parts}
}{}

% 5

\eoce{\qt{Coffee, depression, and physical activity\label{coffee_depression_phys_act}} 
Caffeine is the world's most widely used stimulant, with approximately 80\% consumed 
in the form of coffee. Participants in a study investigating the relationship between 
coffee consumption and exercise were asked to report the number of hours they spent per 
week on moderate (e.g., brisk walking) and vigorous (e.g., strenuous sports and jogging) 
exercise. Based on these data the researchers estimated the total hours of metabolic 
equivalent tasks (MET) per week, a value always greater than 0. The table below gives 
summary statistics of MET for women in this study based on the amount of coffee consumed.
\footfullcite{Lucas:2011}
 
\begin{adjustwidth}{-4em}{-4em}
\begin{center}
\begin{tabular}{l  r  r  r  r  r  r}
                & \multicolumn{5}{c}{\textit{Caffeinated coffee consumption}} \\
\cline{2-6}
                & $\le$ 1 cup/week  & 2-6 cups/week & 1 cup/day 
                                            & 2-3 cups/day  & $\ge$ 4 cups/day  & Total \\
\hline
Mean            & 18.7              & 19.6          & 19.3  
                                            & 18.9          & 17.5 \\
SD              & 21.1              & 25.5          & 22.5  
                                            & 22.0          & 22.0 \\
n               & 12,215            & 6,617         & 17,234    
                                            & 12,290        & 2,383             & 50,739 \\
\hline
\end{tabular}
\end{center}
\end{adjustwidth}

\begin{parts}

\item Write the hypotheses for evaluating if the average physical activity level 
varies among the different levels of coffee consumption.

\item Check conditions and describe any assumptions you must make to proceed with 
the test.

\item Below is part of the output associated with this test. Fill in the empty cells.

\begin{center}
\renewcommand{\arraystretch}{1.25}
\begin{tabular}{lrrrrr}
  \hline
            & Df
                        & Sum Sq
                                    & Mean Sq
                                                & F value
                                                            & Pr($>$F) \\ 
  \hline
coffee      & \fbox{\textcolor{white}{{\footnotesize XXXXX}}}    
                        & \fbox{\textcolor{white}{{\footnotesize XXXXX}}}      
                                    & \fbox{\textcolor{white}{{\footnotesize XXXXX}}}           
                                                & \fbox{\textcolor{white}{{\footnotesize XXXXX}}}
                                                            & 0.0003 \\ 
Residuals   & \fbox{\textcolor{white}{{\footnotesize XXXXX}}} 
                        & 25,564,819
                                    & \fbox{\textcolor{white}{{\footnotesize  XXXXX}}}
                                                &
                                                            &  \\ 
   \hline
Total       & \fbox{\textcolor{white}{{\footnotesize XXXXX}}} 
                        & 25,575,327
\end{tabular}
\end{center}

\item What is the conclusion of the test?

\end{parts}
}{}

% 6

\eoce{\qt{Student performance across discussion sections\label{student_performance_sections}} A professor who teaches a large introductory statistics class (197 students) with eight discussion sections would like to test if student performance differs by discussion section, where each discussion section has a different teaching assistant. The summary table below shows the average final exam score for each discussion section as well as the standard deviation of scores and the number of students in each section.
\begin{center}
\begin{tabular}{rrrrrrrrr}
  \hline
            & Sec 1 & Sec 2 & Sec 3 & Sec 4 & Sec 5 & Sec 6 & Sec 7 & Sec 8 \\ 
  \hline
$n_i$       & 33 & 19 & 10 & 29 & 33 & 10 & 32 & 31 \\ 
$\bar{x}_i$ & 92.94 & 91.11 & 91.80 & 92.45 & 89.30 & 88.30 & 90.12 & 93.35 \\ 
$s_i$       & 4.21 & 5.58 & 3.43 & 5.92 & 9.32 & 7.27 & 6.93 & 4.57 \\ 
   \hline
\end{tabular}
\end{center}
The ANOVA output below can be used to test for differences between the average scores from the different discussion sections.
\begin{center}
\begin{tabular}{lrrrrr}
\hline
            & Df        & Sum Sq & Mean Sq  & F value & Pr($>$F) \\ 
\hline
section         & 7         & 525.01    & 75.00         & 1.87  & 0.0767 \\ 
Residuals   & 189   & 7584.11   & 40.13         &       &  \\ 
\hline
\end{tabular}
\end{center}
Conduct a hypothesis test to determine if these data provide convincing evidence that the average score varies across some (or all) groups. Check conditions and describe any assumptions you must make to proceed with the test.
}{}

% 7

\eoce{\qt{GPA and major\label{gpa_major}} Undergraduate students taking an introductory statistics course at Duke University conducted a survey about GPA and major. The side-by-side box plots show the distribution of GPA among three groups of majors. Also provided is the ANOVA output.
\begin{center}
\includegraphics[width=0.85\textwidth]{ch_inference_for_means/figures/eoce/gpa_major/gpa_major.pdf}
\end{center}
\begin{center}
\begin{tabular}{lrrrrr}
  \hline
            & Df    & Sum Sq    & Mean Sq   & F value   & Pr($>$F) \\ 
  \hline
major       & 2     & 0.03      & 0.015      & 0.185     & 0.8313 \\ 
Residuals   & 195   & 15.77     & 0.081      &           &  \\ 
   \hline
\end{tabular}
\end{center}
\begin{parts}
\item Write the hypotheses for testing for a difference between average GPA across majors.
\item What is the conclusion of the hypothesis test?
\item How many students answered these questions on the survey, i.e. what is the sample size?
\end{parts}
}{}

% 8

\eoce{\qt{Work hours and education\label{work_hours_education}} The General Social Survey 
collects data on demographics, education, and work, among many other characteristics 
of US residents. \footfullcite{data:gss} Using ANOVA, we can consider 
educational attainment levels for all 1,172 respondents at once. Below are the 
distributions of hours worked by educational attainment and relevant summary 
statistics that will be helpful in carrying out this analysis.
\begin{center}

\begin{tabular}{l  r  r  r  r  r  r}
                & \multicolumn{5}{c}{\textit{Educational attainment}} \\
\cline{2-6}
                & Less than HS  & HS    & Jr Coll   & Bachelor's & Graduate & Total \\
\hline
Mean            & 38.67         & 39.6  & 41.39     & 42.55     & 40.85     & 40.45 \\
SD              & 15.81         & 14.97 & 18.1      & 13.62     & 15.51     & 15.17 \\
n               & 121           & 546   & 97        & 253       & 155       & 1,172 \\
\hline
\end{tabular}

\includegraphics[width=\textwidth]{ch_inference_for_means/figures/eoce/work_hours_education/work_hours_education.pdf}
\end{center}
\begin{parts}
\item Write hypotheses for evaluating whether the average number of hours 
worked varies across the five groups.
\item Check conditions and describe any assumptions you must make to proceed 
with the test.
\item Below is part of the output associated with this test. Fill in the 
empty cells.

\begin{center}
\renewcommand{\arraystretch}{1.25}
\begin{tabular}{lrrrrr}
  \hline
            & Df    
                    & Sum Sq        
                            & Mean Sq       
                                    & F-value      
                                            & Pr($>$F) \\ 
  \hline
degree      & \fbox{\textcolor{white}{{\footnotesize XXXXX}}}       
                    & \fbox{\textcolor{white}{{\footnotesize XXXXX}}}       
                            & 501.54    
                                    & \fbox{\textcolor{white}{{\footnotesize XXXXX}}}   
                                            & 0.0682 \\ 
Residuals   & \fbox{\textcolor{white}{{\footnotesize XXXXX}}} 
                    & 267,382     
                            & \fbox{\textcolor{white}{{\footnotesize  XXXXX}}}          
                                    &       
                                            &  \\ 
   \hline
Total       & \fbox{\textcolor{white}{{\footnotesize XXXXX}}} 
                    &\fbox{\textcolor{white}{{\footnotesize XXXXX}}}
\end{tabular}
\end{center}

\item What is the conclusion of the test?

\end{parts}
}{}

% 9

\eoce{\qt{True / False: ANOVA, Part I\label{tf_anova_1}} Determine if the following statements are true or false in ANOVA, and explain your reasoning for statements you identify as false.
\begin{parts}
\item As the number of groups increases, the modified significance level for pairwise tests increases as well.
\item As the total sample size increases, the degrees of freedom for the residuals increases as well.
\item The constant variance condition can be somewhat relaxed when the sample sizes are relatively consistent across groups.
\item The independence assumption can be relaxed when the total sample size is large.
\end{parts}
}{}

% 10

\eoce{\qt{Child care hours\label{child_care_hours}} The China Health and Nutrition 
Survey aims to examine the effects of the health, nutrition, and family planning 
policies and programs implemented by national and local governments.\footfullcite{data:china} It, for example, collects information on number of hours Chinese parents spend 
taking care of their children under age 6. The side-by-side box plots below 
show the distribution of this variable by educational attainment of the parent. 
Also provided below is the ANOVA output for comparing average hours across 
educational attainment categories.
\begin{center}
\includegraphics[width=\textwidth]{ch_inference_for_means/figures/eoce/child_care_hours/child_care_hours.pdf}
\end{center}
\begin{center}
\begin{tabular}{lrrrrr}
\hline
            & Df    & Sum Sq    & Mean Sq   & F value   & Pr($>$F) \\ 
\hline
education   & 4     & 4142.09   & 1035.52   & 1.26      & 0.2846 \\ 
Residuals   & 794   & 653047.83 & 822.48    &           &  \\ 
\hline
\end{tabular}
\end{center}
\begin{parts}
\item Write the hypotheses for testing for a difference between the average 
number of hours spent on child care across educational attainment levels.
\item What is the conclusion of the hypothesis test?
\end{parts}
}{}

% 11

\eoce{\qt{Prison isolation experiment, Part II\label{prison_isolation_anova}} Exercise~\ref{prison_isolation_T} introduced an experiment that was conducted with the goal of identifying a treatment that reduces subjects' psychopathic deviant T scores, where this score measures a person's need for control or his rebellion against control. In Exercise~\ref{prison_isolation_T} you evaluated the success of each treatment individually. An alternative analysis involves comparing the success of treatments. The relevant ANOVA output is given below.
\begin{center}
\begin{tabular}{lrrrrr}
  \hline
 & Df & Sum Sq & Mean Sq & F value & Pr($>$F) \\ 
  \hline
treatment & 2 & 639.48 & 319.74 & 3.33 & 0.0461 \\ 
  Residuals & 39 & 3740.43 & 95.91 &  &  \\ 
   \hline
\multicolumn{6}{r}{$s_{pooled} = 9.793$ on $df=39$}
\end{tabular}
\end{center}
\begin{parts}
\item What are the hypotheses?
\item What is the conclusion of the test? Use a 5\% significance level.
\item If in part~(b) you determined that the test is significant, conduct pairwise tests to determine which groups are different from each other. If you did not reject the null hypothesis in part~(b), recheck your answer.
\end{parts}
}{}

% 12

\eoce{\qt{True / False: ANOVA, Part II\label{tf_anova_2}} Determine if the following statements are true or false, and explain your reasoning for statements you identify as false.

If the null hypothesis that the means of four groups are all the same is rejected using ANOVA at a 5\% significance level, then ...
\begin{parts}
\item we can then conclude that all the means are different from one another.
\item the standardized variability between groups is higher than the standardized variability within groups.
\item the pairwise analysis will identify at least one pair of means that are significantly different.
\item the appropriate $\alpha$ to be used in pairwise comparisons is 0.05 / 4 = 0.0125 since there are four groups.
\end{parts}
}{}

% 13

\eoce{\qt{Gaming and distracted eating, Part I\label{gaming_distracted_eating_intake}}
A group of researchers are interested in the possible effects of distracting 
stimuli during eating, such as an increase or decrease in the amount of food 
consumption. To test this hypothesis, they monitored food intake for a group of 
44 patients who were randomized into two equal groups. The treatment group ate 
lunch while playing solitaire, and the control group ate lunch without any 
added distractions. Patients in the treatment group ate 52.1 grams of biscuits, 
with a standard deviation of 45.1 grams, and patients in the control group ate 
27.1 grams of biscuits, with a standard deviation of 26.4 grams. Do these data 
provide convincing evidence that the average food intake (measured in amount of 
biscuits consumed) is different for the patients in the treatment group? Assume 
that conditions for inference are satisfied. \footfullcite{Oldham:2011}
}{}

% 14

\eoce{\qt{Gaming and distracted eating, Part II\label{gaming_distracted_eating_recall}} 
The researchers from Exercise~\ref{gaming_distracted_eating_intake} also 
investigated the effects of being distracted by a game on how much people eat. 
The 22 patients in the treatment group who ate their lunch while playing 
solitaire were asked to do a serial-order recall of the food lunch items they 
ate. The average number of items recalled by the patients in this group was 4.
9, with a standard deviation of 1.8. The average number of items recalled by 
the patients in the control group (no distraction) was 6.1, with a standard 
deviation of 1.8. Do these data provide strong evidence that the average number 
of food items recalled by the patients in the treatment and control groups are 
different?
}{}

% 15

\eoce{\qt{Sample size and pairing\label{sample_size_pairing}} Determine if the 
following statement is true or false, and if false, explain your reasoning: If 
comparing means of two groups with equal sample sizes, always use a paired test.
}{}

% 16

\eoce{\qt{College credits\label{college_credits}} A college counselor is interested in 
estimating how many credits a student typically enrolls in each semester. The 
counselor decides to randomly sample 100 students by using the registrar's 
database of students. The histogram below shows the distribution of the number 
of credits taken by these students. Sample statistics for this distribution are 
also provided.\\
\begin{minipage}[c]{0.1\textwidth}
\ 
\end{minipage}
\begin{minipage}[c]{0.5\textwidth}
\begin{center}
\includegraphics[width=\textwidth]{ch_inference_for_means/figures/eoce/college_credits/college_credits_hist.pdf}
\end{center}
\end{minipage}
\begin{minipage}[c]{0.32\textwidth}
\begin{center}
\begin{tabular}{l|r l}
Min     & 8 \\
Q1      & 13 \\
Median  & 14 \\
Mean    & 13.65 \\
SD      & 1.91 \\
Q3      & 15 \\
Max     & 18 \\
\end{tabular}
\end{center}
\end{minipage}
\begin{parts}
\item What is the point estimate for the average number of credits taken per 
semester by students at this college? What about the median?
\item What is the point estimate for the standard deviation of the number of 
credits taken per semester by students at this college? What about the IQR?
\item Is a load of 16 credits unusually high for this college? What about 18 
credits? Explain your reasoning. \textit{Hint:} Observations farther than two 
standard deviations from the mean are usually considered to be unusual.
\item The college counselor takes another random sample of 100 students and this 
time finds a sample mean of 14.02 units. Should she be surprised that this 
sample statistic is slightly different than the one from the original sample? 
Explain your reasoning.
\item The sample means given above are point estimates for the mean number of 
credits taken by all students at that college. What measures do we use to 
quantify the variability of this estimate (Hint: recall that 
$SD_{\bar{x}} = \frac{\sigma}{\sqrt{n}}$)? Compute this quantity using the data 
from the original sample.
\end{parts}
}{}

% 17

\eoce{\qt{Hen eggs\label{hen_eggs}} The distribution of the number of eggs laid 
by a certain species of hen during their breeding period has a mean of 35 eggs 
with a standard deviation of 18.2. Suppose a group of researchers 
randomly samples 45 hens of this species, counts the number of eggs laid 
during their breeding period, and records the sample mean. They repeat 
this 1,000 times, and build a distribution of sample 
means. 
\begin{parts}
\item What is this distribution called? 
\item Would you expect the shape of this distribution to be symmetric, right 
skewed, or left skewed? Explain your reasoning.
\item Calculate the variability of this distribution and state the appropriate 
term used to refer to this value.
\item Suppose the researchers' budget is reduced and they are only able to 
collect random samples of 10 hens. The sample mean of the number of eggs is 
recorded, and we repeat this 1,000 times, and build a new distribution of sample 
means. How will the variability of this new distribution compare to the 
variability of the original distribution?
\end{parts}
}{}

% 18

\eoce{\qt{Art after school\label{art_after_school}} Elijah and Tyler, 
two high school juniors, conducted a survey on 15 students at their school, 
asking the students whether they would like the school to offer an 
after-school art program, counted the number of ``yes" answers, and 
recorded the sample proportion. 14 out of the 15 students responded 
``yes". They repeated this 100 times and built a distribution of 
sample means. (Note that this question requires having reviewed
Section~\ref{normalApproxBinomialDistSubsection}
on normal approximation to the binomial distribution.) 
\begin{parts}
\item What is this distribution called? 
\item Would you expect the shape of this distribution to be symmetric, right 
skewed, or left skewed? Explain your reasoning.
\item Calculate the variability of this distribution and state the appropriate 
term used to refer to this value.
\item Suppose that the students were able to recruit a few more friends to help 
them with sampling, and are now able to collect data from random samples of 25 
students. Once again, they record the number of ``yes" answers, and record the 
sample proportion, and repeat this 100 times to build a new distribution of 
sample proportions. How will the variability of this new distribution compare to the variability of the original distribution?
\end{parts}
}{}

% 19

\eoce{\qt{Exclusive relationships\label{exclusive_relationships}} A survey conducted 
on a reasonably random sample of 203 undergraduates asked, among many other 
questions, about the number of exclusive relationships these students have been 
in. The histogram below shows the distribution of the data from this sample. 
The sample average is 3.2 with a standard deviation of 1.97.
\begin{center}
\includegraphics[width=0.6\textwidth]{ch_inference_for_means/figures/eoce/exclusive_relationships/exclusive_relationships_rel_hist.pdf}
\end{center}
Estimate the average number of exclusive relationships Duke students have been 
in using a 90\% confidence interval and interpret this interval in context. 
Check any conditions required for inference, and note any assumptions you must 
make as you proceed with your calculations and conclusions.
}{}

% 20

\eoce{\qt{Age at first marriage, Part I\label{age_at_first_marriage_intro}} 
The National Survey of Family Growth conducted by the Centers for Disease 
Control gathers information on family life, marriage and divorce, pregnancy, 
infertility, use of contraception, and men's and women's health. One of the 
variables collected on this survey is the age at first marriage. The histogram 
below shows the distribution of ages at first marriage of 5,534 randomly sampled 
women between 2006 and 2010. The average age at first marriage among these women 
is 23.44 with a standard deviation of 4.72.\footfullcite{data:nsfg:2010}
\begin{center}
\includegraphics[width=0.6\textwidth]{ch_inference_for_means/figures/eoce/age_at_first_marriage_intro/age_at_first_marriage_intro_hist.pdf}
\end{center}
Estimate the average age at first marriage of women using a 95\% confidence 
interval, and interpret this interval in context. Discuss any relevant 
assumptions.
}{}

% 21

\eoce{\qt{Online communication\label{online_communication}} A study suggests that the 
average college student spends 10 hours per week communicating with others 
online. You believe that this is an underestimate and decide to collect your 
own sample for a hypothesis test. You randomly sample 60 students from your 
dorm and find that on average they spent 13.5 hours a week communicating with 
others online. A friend of yours, who offers to help you with the hypothesis 
test, comes up with the following set of hypotheses. Indicate any errors you see.
\begin{align*}
H_0&: \bar{x} < 10~hours \\
H_A&: \bar{x} > 13.5~hours
\end{align*}
}{}

% 22

\eoce{\qt{Age at first marriage, Part II\label{age_at_first_marriage_hyp_errors}} Exercise~\ref{age_at_first_marriage_intro} presents the results 
of a 2006 - 2010 survey showing that the average age of women at first marriage 
is 23.44. Suppose a social scientist believes that this value has increased 
since then, but she would also be interested if she found a decrease. Below is how she 
set up her hypotheses. Indicate any errors you see.
\begin{align*}
H_0&: \bar{x} = 23.44~years~old \\
H_A&: \bar{x} > 23.44~years~old
\end{align*}
}{}

% 23

\eoce{\qt{TITLE\label{ID}}
\textbf{\color{red}WRITE NEW ANOVA EXERCISE}
\begin{parts}
\item Part (a).
\item Part (b).
\item Part (c).
\end{parts}
}{}
