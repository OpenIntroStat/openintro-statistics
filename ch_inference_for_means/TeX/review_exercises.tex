\reviewexercisesheader{}

% 47

\eoce{\qt{Gaming and distracted eating, Part I\label{gaming_distracted_eating_intake}}
A group of researchers are interested in the possible effects of distracting 
stimuli during eating, such as an increase or decrease in the amount of food 
consumption. To test this hypothesis, they monitored food intake for a group of 
44 patients who were randomized into two equal groups. The treatment group ate 
lunch while playing solitaire, and the control group ate lunch without any 
added distractions. Patients in the treatment group ate 52.1 grams of biscuits, 
with a standard deviation of 45.1 grams, and patients in the control group ate 
27.1 grams of biscuits, with a standard deviation of 26.4 grams. Do these data 
provide convincing evidence that the average food intake (measured in amount of 
biscuits consumed) is different for the patients in the treatment group? Assume 
that conditions for inference are satisfied. \footfullcite{Oldham:2011}
}{}

% 48

\eoce{\qt{Gaming and distracted eating, Part II\label{gaming_distracted_eating_recall}} 
The researchers from Exercise~\ref{gaming_distracted_eating_intake} also 
investigated the effects of being distracted by a game on how much people eat. 
The 22 patients in the treatment group who ate their lunch while playing 
solitaire were asked to do a serial-order recall of the food lunch items they 
ate. The average number of items recalled by the patients in this group was 4.
9, with a standard deviation of 1.8. The average number of items recalled by 
the patients in the control group (no distraction) was 6.1, with a standard 
deviation of 1.8. Do these data provide strong evidence that the average number 
of food items recalled by the patients in the treatment and control groups are 
different?
}{}

% 49

\eoce{\qt{Sample size and pairing\label{sample_size_pairing}} Determine if the 
following statement is true or false, and if false, explain your reasoning: If 
comparing means of two groups with equal sample sizes, always use a paired test.
}{}

% 50

\eoce{\qt{College credits\label{college_credits}}
A college counselor is interested in 
estimating how many credits a student typically enrolls
in each semester.
The counselor decides to randomly sample 100 students
by using the registrar's 
database of students.
The histogram below shows the distribution of the number 
of credits taken by these students.
Sample statistics for this distribution are 
also provided.\\
\begin{minipage}[c]{0.1\textwidth}
\ 
\end{minipage}
\begin{minipage}[c]{0.5\textwidth}
\begin{center}
\FigureFullPath[A histogram is shown for "Number of credits". The distribution is centered at about 13 and is very roughly bell-shaped with data ranging from 8 to 18 with no apparent outliers.]{}{ch_inference_for_means/figures/eoce/college_credits/college_credits_hist}
\end{center}
\end{minipage}
\begin{minipage}[c]{0.32\textwidth}
\begin{center}
\begin{tabular}{l|r l}
Min     & 8 \\
Q1      & 13 \\
Median  & 14 \\
Mean    & 13.65 \\
SD      & 1.91 \\
Q3      & 15 \\
Max     & 18 \\
\end{tabular}
\end{center}
\end{minipage}
\begin{parts}
\item What is the point estimate for the average
number of credits taken per semester by students at this college?
What about the median?
\item What is the point estimate for the standard deviation
of the number of credits taken per semester by students at
this college?
What about the IQR?
\item Is a load of 16 credits unusually high for this college?
What about 18 credits?
Explain your reasoning.
\item The college counselor takes another
random sample of 100 students and this 
time finds a sample mean of 14.02 units.
Should she be surprised that this sample
statistic is slightly different than the
one from the original sample? 
Explain your reasoning.
\item
The sample means given above are point estimates
for the mean number of 
credits taken by all students at that college.
What measures do we use to 
quantify the variability of this estimate?
Compute this quantity using the data 
from the original sample.
\end{parts}
}{}

\D{\newpage}

% 51

\eoce{\qt{Hen eggs\label{hen_eggs}} The distribution of the number of eggs laid 
by a certain species of hen during their breeding period has a mean of 35 eggs 
with a standard deviation of 18.2. Suppose a group of researchers 
randomly samples 45 hens of this species, counts the number of eggs laid 
during their breeding period, and records the sample mean. They repeat 
this 1,000 times, and build a distribution of sample 
means. 
\begin{parts}
\item What is this distribution called? 
\item Would you expect the shape of this distribution to be symmetric, right 
skewed, or left skewed? Explain your reasoning.
\item Calculate the variability of this distribution and state the appropriate 
term used to refer to this value.
\item Suppose the researchers' budget is reduced and they are only able to 
collect random samples of 10 hens. The sample mean of the number of eggs is 
recorded, and we repeat this 1,000 times, and build a new distribution of sample 
means. How will the variability of this new distribution compare to the 
variability of the original distribution?
\end{parts}
}{}

% 52

\eoce{\qt{Forest management\label{forest_mgmt_tree_growth}}
Forest rangers wanted to better understand the rate
of growth for younger trees in the park.
They took measurements of a random sample of 50 young trees
in 2009 and again measured those same trees in 2019.
The data below summarize their measurements,
where the heights are in feet:
\begin{center}
\begin{tabular}{l c c c}
\hline
          & 2009   & 2019  & Differences\\
\hline  
$\bar{x}$ & 12.0  & 24.5  & 12.5 \\
$s$       & 3.5   & 9.5   & 7.2 \\
$n$       & 50    & 50    & 50 \\
\hline
\end{tabular}
\end{center}
Construct a 99\% confidence interval for the
average growth of (what had been) younger trees
in the park over 2009-2019.
}{}

% 53

\eoce{\qt{Experiment resizing\label{tech_exp_resizing}}
At a startup company running a new weather app, an engineering
team generally runs experiments where a random sample of 1\%
of the app's visitors in the control group and another
1\% were in the treatment group to test each new feature.
The team's core goal is to increase a metric
called \emph{daily visitors},
which is essentially the number of visitors to the app
each day.
They track this metric in each experiment arm and
as their core experiment metric.
In their most recent experiment, the team tested
including a new animation when the app started,
and the number of daily visitors in this experiment
stabilized at +1.2\% with a 95\% confidence interval
of (-0.2\%, +2.6\%).
This means if this new app start animation was launched,
the team thinks they might lose as many as 0.2\% of daily
visitors or gain as many as 2.6\% more daily visitors.
Suppose you are consulting as the team's data scientist,
and after discussing with the team,
you and they agree that they should run
another experiment that is bigger.
You also agree that this new experiment
should be able to detect a gain in the daily visitors
metric of 1.0\% or more with 80\% power.
Now they turn to you and ask,
``How big of an experiment do we need to run
to ensure we can detect this effect?''
\begin{parts}
\item\label{tech_exp_resizing_target_se}
    How small must the standard error be if
    the team is to be able to detect an effect
    of 1.0\% with 80\% power and a significance
    level of $\alpha = 0.05$?
    You may safely assume the percent change in
    daily visitors metric follows a normal distribution.
\item\label{tech_exp_resizing_original_se}
    Consider the first experiment, where
    the point estimate was +1.2\% and the
    95\% confidence interval was (-0.2\%, +2.6\%).
    If that point estimate followed a normal
    distribution, what was the standard error
    of the estimate?
\item\label{tech_exp_resizing_ratio}
    The ratio of the standard error from
    part~(\ref{tech_exp_resizing_target_se})
    vs the standard error from
    part~(\ref{tech_exp_resizing_original_se})
    should be~1.97.
    How much bigger of an experiment is needed
    to shrink a standard error by a factor of~1.97?
\item
    Using your answer from
    part~(\ref{tech_exp_resizing_ratio})
    and that the original experiment was
    a 1\% vs 1\% experiment to recommend
    an experiment size to the team.
\end{parts}
}{}

\D{\newpage}

% 54

\eoce{\qt{Torque on a rusty bolt\label{torque_on_rusty_bolt}}
Project Farm is a YouTube channel that routinely
compares different products.
In one episode, the channel evaluated different
options for loosening rusty
bolts.\footfullcite{youtube:torque_on_rusty_bolt}
Eight options were evaluated,
including a control group where no treatment was given
(``none'' in the graph),
to determine which was most effective.
For all treatments, there were four bolts tested,
except for a treatment of heat with a blow torch,
where only two data points were collected.
The results are shown in the figure below:
\begin{center}
\FigureFullPath[A side-by-side dot plot is shown for "Torque required to loosen a rusty bolt, in foot-pounds". There are only 2 to 4 observations per option, which are roughly as follows: Heat (82, 98), WD-40 (106, 118, 129, 131), Royal Purple (108, 114, 122, 132), PB Blaster (110, 124, 127, 128), Liquid Wrench (85, 88, 98, 114), AeroKroil (107, 125, 132, 134), Acetone/ATF (105, 107, 114, 129), and "none" (110, 123, 129, 142).)]{0.8}{ch_inference_for_means/figures/eoce/torque_on_rusty_bolt/torque_on_rusty_bolt_dot_plot}
\end{center}
\begin{parts}
\item\label{torque_on_rusty_bolt_appropriate}
    Do you think it is reasonable to apply ANOVA in this case?
\item
    Regardless of your answer in
    part~(\ref{torque_on_rusty_bolt_appropriate}),
    describe hypotheses for ANOVA in this context,
    and use the table below to carry out the test.
    Give your conclusion in the context of the data.
    \begin{center}
    \begin{tabular}{lrrrrr}
    \hline
    & Df & Sum Sq & Mean Sq & F value & Pr($>$F) \\ 
    \hline
    treatment & 7 & 3603.43 & 514.78 & 4.03 & 0.0056 \\ 
    Residuals & 22 & 2812.80 & 127.85 &  &  \\ 
    \hline
    \end{tabular}
    \end{center}
\item\label{torque_on_rusty_bolt_pvalues}
    The table below are p-values for pairwise $t$-tests
    comparing each of the different groups.
    These p-values have not been corrected for multiple
    comparisons.
    Which pair of groups appears most likely to represent
    a difference?
    \begin{center}\footnotesize
    \begin{tabular}{l ccc ccc c}
    \hline
    & AeroKroil & Heat & Liquid Wrench & none &
        PB Blaster & Royal Purple & WD-40 \\ 
    \hline
    Acetone/ATF & 0.2026 & 0.0308 & 0.0476 & 0.1542 &
        0.3294 & 0.5222 & 0.3744 \\ 
    AeroKroil &  & 0.0027 & 0.0025 & 0.8723 & 0.7551 &
        0.5143 & 0.6883 \\ 
    Heat &  &  & 0.5580 & 0.0020 & 0.0050 & 0.0096 &
        0.0059 \\ 
    Liquid Wrench &  &  &  & 0.0017 & 0.0053 &
        0.0117 & 0.0065 \\ 
    none &  &  &  &  & 0.6371 & 0.4180 & 0.5751 \\ 
    PB Blaster &  &  &  &  &  & 0.7318 & 0.9286 \\ 
    Royal Purple &  &  &  &  &  &  & 0.8000 \\
    \hline
    \end{tabular}
    \end{center}
\item
    There are 28 p-values shown in the table in
    part~(\ref{torque_on_rusty_bolt_pvalues}).
    Determine if any of them are statistically
    significant after correcting for multiple
    comparisons.
    If so, which one(s)?
    Explain your answer.
\end{parts}
}{}

% 55

\eoce{\qt{Exclusive relationships\label{exclusive_relationships}} A survey conducted 
on a reasonably random sample of 203 undergraduates asked, among many other 
questions, about the number of exclusive relationships these students have been 
in. The histogram below shows the distribution of the data from this sample. 
The sample average is 3.2 with a standard deviation of 1.97.
\begin{center}
\FigureFullPath[A histogram is shown for "Number of exclusive relationships". The distribution has a peak between 1 and 2 of about 101, a substantial dip for the 2 to 3 bin at a value of about 2, and the 3 to 4 bin is about 50, 4 to 5 bin a value of about 25, and the data continues to taper off with a maximum value of "10" shown.]{0.6}{ch_inference_for_means/figures/eoce/exclusive_relationships/exclusive_relationships_rel_hist}
\end{center}
Estimate the average number of exclusive relationships Duke students have been 
in using a 90\% confidence interval and interpret this interval in context. 
Check any conditions required for inference, and note any assumptions you must 
make as you proceed with your calculations and conclusions.
}{}

% 56

\eoce{\qt{Age at first marriage, Part I\label{age_at_first_marriage_intro}} 
The National Survey of Family Growth conducted by the Centers for Disease 
Control gathers information on family life, marriage and divorce, pregnancy, 
infertility, use of contraception, and men's and women's health. One of the 
variables collected on this survey is the age at first marriage. The histogram 
below shows the distribution of ages at first marriage of 5,534 randomly sampled 
women between 2006 and 2010. The average age at first marriage among these women 
is 23.44 with a standard deviation of 4.72.\footfullcite{data:nsfg:2010}
\begin{center}
\FigureFullPath[A histogram is shown for "Age at first marriage". The distribution is right-skewed, centered at about 23, has a standard deviation of about 5. The data smoothly tapers off in each direction but do not extend below about 12 or above 45.]{0.6}{ch_inference_for_means/figures/eoce/age_at_first_marriage_intro/age_at_first_marriage_intro_hist}
\end{center}
Estimate the average age at first marriage of women using a 95\% confidence 
interval, and interpret this interval in context. Discuss any relevant 
assumptions.
}{}

% 57

\eoce{\qt{Online communication\label{online_communication}} A study suggests that the 
average college student spends 10 hours per week communicating with others 
online. You believe that this is an underestimate and decide to collect your 
own sample for a hypothesis test. You randomly sample 60 students from your 
dorm and find that on average they spent 13.5 hours a week communicating with 
others online. A friend of yours, who offers to help you with the hypothesis 
test, comes up with the following set of hypotheses. Indicate any errors you see.
\begin{align*}
H_0&: \bar{x} < 10~hours \\
H_A&: \bar{x} > 13.5~hours
\end{align*}
}{}

% 58

\eoce{\qt{Age at first marriage, Part II\label{age_at_first_marriage_hyp_errors}} Exercise~\ref{age_at_first_marriage_intro} presents the results 
of a 2006 - 2010 survey showing that the average age of women at first marriage 
is 23.44.
Suppose a social scientist thinks this value has changed 
since the survey was taken.
Below is how she set up her hypotheses.
Indicate any errors you see.
\begin{align*}
H_0&: \bar{x} \neq 23.44~years~old \\
H_A&: \bar{x} = 23.44~years~old
\end{align*}
}{}
