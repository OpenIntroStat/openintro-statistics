\exercisesheader{}

% 23

\eoce{\qt{Friday the 13$^{\text{th}}$, Part I\label{friday_13th_traffic}} In the 
early 1990's, researchers in the UK collected data on traffic flow, number of 
shoppers, and traffic accident related emergency room admissions on Friday the 
13$^{\text{th}}$ and the previous Friday, Friday the 6$^{\text{th}}$. The 
histograms below show the distribution of number of cars passing by a specific 
intersection on Friday the 6$^{\text{th}}$ and Friday the 13$^{\text{th}}$ for 
many such date pairs. Also given are some sample statistics, where the 
difference is the number of cars on the 6th minus the number of cars on the 13th.\footfullcite{Scanlon:1993}
\begin{center}
\FigureFullPath[Three histograms are shown. The first histogram is for "Friday the 6th", which has values ranging from 110,000 to 140,000. The second histogram is for "Friday the 13th", which also has values ranging from 110,000 to 140,000. The third histogram is for "Difference", with values ranging from 0 to 5,000. While the first two distributions are relatively uniform across the range, the last distribution has most of its distribution ranging between 0 and 3,000, with one observation in the 4,000 to 5,000 bin, which represents one value.]{}{ch_inference_for_means/figures/eoce/friday_13th_traffic/friday_13th_traffic_hist} \\
$\:$ \\
{\small
\begin{tabular}{l c c c}
\hline
        & 6$^{\text{th}}$   & 13$^{\text{th}}$  & Diff.\\
\hline  
$\bar{x}$   &128,385            & 126,550       & 1,835 \\
$s$     &7,259          & 7,664         & 1,176 \\
$n$     &10             & 10                & 10 \\
\hline
\end{tabular}
}
\end{center}
\begin{parts}
\item Are there any underlying structures in these data that should be 
considered in an analysis? Explain.
\item What are the hypotheses for evaluating whether the number of people out 
on Friday the 6$^{\text{th}}$ is different than the number out on Friday the 
13$^{\text{th}}$?
\item Check conditions to carry out the hypothesis test from part~(b).
\item Calculate the test statistic and the p-value.
\item What is the conclusion of the hypothesis test?
\item Interpret the p-value in this context.
\item What type of error might have been made in the conclusion of your test? 
Explain.
\end{parts}
}{}

% 24

\eoce{\qt{Diamonds, Part I\label{diamonds_1}} Prices of diamonds are determined by 
what is known as the 4 Cs: cut, clarity, color, and carat weight. The prices of 
diamonds go up as the carat weight increases, but the increase is not smooth. 
For example, the difference between the size of a 0.99 carat diamond and a 1 
carat diamond is undetectable to the naked human eye, but the price of a 1 
carat diamond tends to be much higher than the price of a 0.99 diamond. In this 
question we use two random samples of diamonds, 0.99 carats and 1 carat, each 
sample of size 23, and compare the average prices of the diamonds. In order to 
be able to compare equivalent units, we first divide the price for each diamond 
by 100 times its weight in carats. That is, for a 0.99 carat diamond, we divide 
the price by 99. For a 1 carat diamond, we divide the price by 100. The 
distributions and some sample statistics are shown below.\footfullcite{ggplot2} \\[1mm]
\begin{minipage}[c]{0.57\textwidth}
Conduct a hypothesis test to evaluate if there is a difference between the 
average standardized prices of 0.99 and 1 carat diamonds. Make sure to state 
your hypotheses clearly, check relevant conditions, and interpret your results 
in context of the data. \\[2mm]
\begin{tabular}{l c c }
\hline
        & 0.99 carats       & 1 carat\\
\hline  
Mean    & \$44.51          & \$56.81           \\
SD      & \$13.32          &\$16.13            \\
n       &23             & 23 \\
\hline
\end{tabular}
\end{minipage}%
\begin{minipage}[c]{0.43\textwidth}
\begin{center}
\FigureFullPath[Side-by-side box plot for "Point price, in dollars". The two categories shown are for "0.99 carats" and "1 carat" diamonds. The 0.99 carat diamonds have their box running from about \$36 to \$57, a median of about \$49, and the whiskers spanning about \$19 to \$62. The 1 carat diamonds have their box running from about \$48 to \$72, a median of about \$55, and the whiskers spanning about \$34 to \$72.]{0.875}{ch_inference_for_means/figures/eoce/diamonds_1/diamonds_box.pdf}
\end{center}
\end{minipage}
}{}

\D{\newpage}

% 25

\eoce{\qt{Friday the 13$^{\text{th}}$, Part II\label{friday_13th_accident}}
The Friday the $13^{th}$ study reported in
Exercise~\ref{friday_13th_traffic} also provides data on traffic
accident related emergency room admissions.
The distributions of these counts from Friday the 6$^{\text{th}}$ and
Friday the 13$^{\text{th}}$ are shown below for six such paired dates
along with summary statistics.
You may assume that conditions for inference are met.
\begin{center}
\FigureFullPath[Three histograms are shown. The first histogram is for "Friday the 6th", which has values ranging across 3 to 12. The second histogram is for "Friday the 13th", which has values ranging from 4 to 14. The third histogram is for "Difference", with values ranging from -8 to positive 2.]{0.9}{ch_inference_for_means/figures/eoce/friday_13th_accident/friday_13th_accident_hist} \\
$\:$ \\
\begin{minipage}[c]{0.32\textwidth}
\begin{tabular}{l c c c}
\hline
        & 6$^{\text{th}}$   & 13$^{\text{th}}$  & diff\\
\hline  
Mean    &7.5                & 10.83             & -3.33 \\
SD      &3.33           & 3.6               & 3.01 \\
n       &6              & 6             & 6 \\
\hline
\end{tabular}
\end{minipage}
\end{center}

\begin{parts}
\item Conduct a hypothesis test to evaluate if there is a difference between 
the average numbers of traffic accident related emergency room admissions 
between Friday the 6$^{\text{th}}$ and Friday the~13$^{\text{th}}$.
\item Calculate a 95\% confidence interval for the difference between the 
average numbers of traffic accident related emergency room admissions between 
Friday the 6$^{\text{th}}$ and Friday the 13$^{\text{th}}$.
\item The conclusion of the original study states, ``Friday 13th is unlucky for 
some. The risk of hospital admission as a result of a transport accident may be 
increased by as much as 52\%. Staying at home is recommended.'' Do you agree 
with this statement? Explain your reasoning.
\end{parts}
}{}

% 26

\eoce{\qt{Diamonds, Part II\label{diamonds_2}} In Exercise~\ref{diamonds_1}, we 
discussed diamond prices (standardized by weight) for diamonds with weights 0.
99 carats and 1 carat. See the table for summary statistics, and then construct 
a 95\% confidence interval for the average difference between the standardized 
prices of 0.99 and 1 carat diamonds. You may assume the conditions for 
inference are met.
\begin{center}
\begin{tabular}{l c c }
\hline
        & 0.99 carats       & 1 carat\\
\hline  
Mean    & \$44.51          & \$56.81           \\
SD      & \$13.32          &\$16.13            \\
n       &23             & 23 \\
\hline
\end{tabular}
\end{center}
}{}

% 27

\eoce{\qt{Chicken diet and weight,
    Part I\label{chick_wts_linseed_horsebean}}
Chicken farming is a multi-billion dollar industry,
and any methods that increase the growth rate of young
chicks can reduce consumer costs while increasing
company profits, possibly by millions of dollars.
An experiment was conducted to measure and compare
the effectiveness of various feed supplements on the
growth rate of chickens.
Newly hatched chicks were randomly allocated into six groups, 
and each group was given a different feed supplement.
Below are some summary statistics from this data set along
with box plots showing the distribution of weights by
feed type.\footfullcite{data:chickwts}

\noindent\begin{minipage}[c]{0.65\textwidth}
\begin{center}
\FigureFullPath[A side-by-side box plot is shown for "Weight, in grams" for several feed types. The width of the data range for each feed type spans about 150 grams. However, they are centered at different locations: about 325 for "casein", about 150 for "horsebean", about 225 for "linseed", about 275 for "meatmeal", about 250 for "soybean", and about 325 for "sunflower".]{}{ch_inference_for_means/figures/eoce/chick_wts_linseed_horsebean/chick_wts_box.pdf}
\end{center}
\end{minipage}
\begin{minipage}[c]{0.35\textwidth}
{\footnotesize\begin{tabular}{l c c c}
\hline
            & Mean      & SD        & n \\
\hline
casein          & 323.58        & 64.43 & 12 \\
horsebean   & 160.20        & 38.63 & 10 \\
linseed         & 218.75        & 52.24 & 12 \\
meatmeal    & 276.91        & 64.90 & 11 \\
soybean         & 246.43        & 54.13 & 14 \\
sunflower       & 328.92        & 48.84 & 12 \\
\hline
\end{tabular}}
\end{minipage} 

\begin{parts}
\item Describe the distributions of weights of chickens that were fed linseed 
and horsebean.
\item Do these data provide strong evidence that the average weights of 
chickens that were fed linseed and horsebean are different? Use a 5\% 
significance level.
\item What type of error might we have committed? Explain.
\item Would your conclusion change if we used $\alpha = 0.01$?
\end{parts}
}{}

\D{\newpage}

% 28

\eoce{\qt{Fuel efficiency of manual and automatic cars, Part I\label{fuel_eff_city}} 
Each year the US Environmental Protection Agency (EPA)
releases fuel economy data on cars manufactured in that year.
Below are summary statistics on fuel efficiency (in miles/gallon)
from random samples of cars with manual and automatic transmissions.
Do these data provide strong evidence of a difference between the
average fuel efficiency of cars with manual and automatic
transmissions in terms of their average city mileage?
Assume that conditions for inference are
satisfied. \footfullcite{data:epaMPG}

\noindent\begin{minipage}[c]{0.38\textwidth}
\begin{center}
\begin{tabular}{l c c }
\hline
        & \multicolumn{2}{c}{City MPG} \\
\hline
        & Automatic     & Manual         \\
Mean    & 16.12         & 19.85      \\
SD      & 3.58          & 4.51       \\
n       & 26            & 26 \\
\hline
& \\
& \\
\end{tabular}
\end{center}
\end{minipage}
\begin{minipage}[c]{0.6\textwidth}
\begin{center}
\FigureFullPath[A side-by-side box plot is shown for "City MPG" for "automatic" and "manual" cars. The "automatic" box plot has its box spanning approximately 14 to 19, has a median of about 16, and its whiskers extending down to about 7 and up to about 24. The "manual" box plot has its box spanning approximately 18 to 24, has a median of about 21, and its whiskers extending down to about 8 and up to about 31.]{0.7}{ch_inference_for_means/figures/eoce/fuel_eff_city/fuel_eff_city_box.pdf}
\end{center}
\end{minipage}
}{}

% 29

\eoce{\qt{Chicken diet and weight, Part II\label{chick_wts_casein_soybean}} Casein is 
a common weight gain supplement for humans. Does it have an effect on chickens? 
Using data provided in Exercise~\ref{chick_wts_linseed_horsebean}, test the 
hypothesis that the average weight of chickens that were fed casein is 
different than the average weight of chickens that were fed soybean. If your 
hypothesis test yields a statistically significant result, discuss whether or 
not the higher average weight of chickens can be attributed to the casein diet. 
Assume that conditions for inference are satisfied.
}{}

% 30

\eoce{\qt{Fuel efficiency of manual and automatic cars, Part II\label{fuel_eff_hway}} 
The table provides summary statistics on highway fuel economy
of the same 52 cars from Exercise~\ref{fuel_eff_city}.
Use these statistics to calculate a 98\% confidence interval
for the difference between average highway mileage of manual
and automatic cars, and interpret this interval in the context
of the data.\footfullcite{data:epaMPG}

\noindent\begin{minipage}[c]{0.38\textwidth}
\begin{center}
\begin{tabular}{l c c }
\hline
        & \multicolumn{2}{c}{Hwy MPG} \\
\hline
            & Automatic     & Manual         \\
Mean    & 22.92         & 27.88          \\
SD      & 5.29          & 5.01           \\
n       & 26            & 26 \\
\hline
& \\
& \\
\end{tabular}
\end{center}
\end{minipage}
\begin{minipage}[c]{0.6\textwidth}
\begin{center}
\FigureFullPath[A side-by-side box plot is shown for "Highway MPG" for "automatic" and "manual" cars. The "automatic" box plot has its box spanning approximately 20 to 26, has a median of about 23, and its whiskers extending down to about 14 and up to about 34. The "manual" box plot has its box spanning approximately 26 to 32, has a median of about 29, and its whiskers extending down to about 17 and up to about 38.]{0.7}{ch_inference_for_means/figures/eoce/fuel_eff_hway/fuel_eff_hway_box.pdf}
\end{center}
\end{minipage}
}{}

\D{\newpage}

% 31

\eoce{\qt{Prison isolation experiment, Part I\label{prison_isolation_T}}
Subjects from Central Prison in Raleigh, NC, volunteered
for an experiment involving an ``isolation'' experience.
The goal of the experiment was to find a treatment 
that reduces subjects' psychopathic deviant T scores.
This score measures a person's need for control or their rebellion against 
control, and it is part of a commonly used mental health test called the 
Minnesota Multiphasic Personality Inventory (MMPI) test. The experiment had 
three treatment groups: 
\begin{enumerate}[(1)]
\setlength{\itemsep}{0mm}
\item
    Four hours of sensory restriction plus a 15 minute
    ``therapeutic" tape advising that professional help
    is available.
\item
    Four hours of sensory restriction plus a 15 minute
    ``emotionally neutral'' tape on training hunting dogs.
\item
    Four hours of  sensory restriction but no taped message.
\end{enumerate}
Forty-two subjects were randomly assigned to these treatment groups, and an 
MMPI test was administered before and after the treatment. Distributions of the 
differences between pre and post treatment scores (pre - post) are shown below, 
along with some sample statistics. Use this information to independently test 
the effectiveness of each treatment. Make sure to clearly state your 
hypotheses, check conditions, and interpret results in the context of the data.\footfullcite{data:prison}

\begin{center}
\FigureFullPath[Three box plots are shown for Treatments 1, 2, and 3. The box plot for "Treatment 1" is slightly right skewed with values ranging from about -10 to about positive 40, and this distribution has one borderline outlier between 30 and 40. The box plot for "Treatment 2" is about symmetric with values ranging from about -20 to about positive 20. The box plot for "Treatment 3" is left skewed with values ranging from about -30 to about positive 10.]{}{ch_inference_for_means/figures/eoce/prison_isolation_T/prison_isolation_hist} \\
$\:$ \\
\begin{tabular}{l  r  r  r  r  }
\hline
                & Tr 1  & Tr 2  & Tr 3      \\
\hline
Mean            & 6.21  & 2.86  & -3.21           \\
SD              & 12.3  & 7.94  & 8.57       \\
n               & 14        & 14        & 14     \\
\hline
\end{tabular}
\end{center}
}{}

% 32

\eoce{\qt{True / False: comparing means\label{tf_compare_means}} Determine if the 
following statements are true or false, and explain your reasoning for 
statements you identify as false.
\begin{parts}
\item When comparing means of two samples where $n_1 = 20$ and $n_2 = 40$, we 
can use the normal model for the difference in means since $n_2 \ge 30$.
\item As the degrees of freedom increases, the $t$-distribution approaches 
normality.
\item We use a pooled standard error for calculating the standard error of the 
difference between means when sample sizes of groups are equal to each other.
\end{parts}
}{}
