


%_______________
\newpage\subsection*{Exercises} % Geometric distribution

% 1

\eoce{\qtq{Is it Bernoulli\label{is_it_bernouilli}} Determine if each trial can be 
considered an independent Bernoulli trial for the following situations.
\begin{parts}
\item Cards dealt in a hand of poker.
\item Outcome of each roll of a die.
\end{parts}
}{}

% 2

\eoce{\qt{With and without replacement\label{with_without_replacement}} In the 
following situations assume that half of the specified population is male and 
the other half is female.
\begin{parts}
\item Suppose you're sampling from a room with 10 people. What is the 
probability of sampling two females in a row when sampling with replacement? 
What is the probability when sampling without replacement?
\item Now suppose you're sampling from a stadium with 10,000 people. What is 
the probability of sampling two females in a row when sampling with 
replacement? What is the probability when sampling without replacement?
\item We often treat individuals who are sampled from a large population as 
independent. Using your findings from parts~(a) and~(b), explain whether or 
not this assumption is reasonable.
\end{parts}
}{}

% 3

\eoce{\qt{Married women} \label{married_women} The American Community Survey 
estimates that 47.1\% of women ages 15 years and over are married.
\footfullcite{marWomenACS}
\begin{parts}
\item We randomly select three women between these ages. What is the 
probability that the third woman selected is the only one who is married?
\item What is the probability that all three randomly selected women are 
married?
\item On average, how many women would you expect to sample before selecting 
a married woman? What is the standard deviation?
\item If the proportion of married women was actually 30\%, how many women 
would you expect to sample before selecting a married woman? What is the 
standard deviation?
\item Based on your answers to parts (c) and (d), how does decreasing the 
probability of an event affect the mean and standard deviation of the wait 
time until success?
\end{parts}
}{}

% 4

\eoce{\qt{Defective rate\label{defective_rate}} A machine that produces a special 
type of transistor (a component of computers) has a 2\% defective rate. The 
production is considered a random process where each transistor is 
independent of the others.
\begin{parts}
\item What is the probability that the $10^{th}$ transistor produced is the 
first with a defect?
\item What is the probability that the machine produces no defective 
transistors in a batch of 100?
\item On average, how many transistors would you expect to be produced before 
the first with a defect? What is the standard deviation?
\item Another machine that also produces transistors has a 5\% defective rate 
where each transistor is produced independent of the others. On average how 
many transistors would you expect to be produced with this machine before the 
first with a defect? What is the standard deviation?
\item Based on your answers to parts (c) and (d), how does increasing the 
probability of an event affect the mean and standard deviation of the wait 
time until success?
\end{parts}
}{}

% 5

\eoce{\qt{Eye color, Part I\label{eye_color_geometric}} A husband and wife both 
have brown eyes but carry genes that make it possible for their children to 
have brown eyes (probability 0.75), blue eyes (0.125), or green eyes (0.125).
\begin{parts}
\item What is the probability the first blue-eyed child they have is their 
third child? Assume that the eye colors of the children are independent of 
each other.
\item On average, how many children would such a pair of parents have before 
having a blue-eyed child? What is the standard deviation of the number of 
children they would expect to have until the first blue-eyed child?
\end{parts}
}{}

% 6

\eoce{\qt{Speeding on the I-5, Part II\label{speeding_i5_geometric}} 
Exercise~\ref{speeding_i5_intro} states that the distribution of speeds of 
cars traveling on the Interstate 5 Freeway (I-5) in California is nearly 
normal with a mean of 72.6 miles/hour and a standard deviation of 4.78 
miles/hour. The speed limit on this stretch of the I-5 is 70 miles/hour.
\begin{parts}
\item A highway patrol officer is hidden on the side of the freeway. What is 
the probability that 5~cars pass and none are speeding? Assume that the 
speeds of the cars are independent of each other.
\item On average, how many cars would the highway patrol officer expect to 
watch until the first car that is speeding? What is the standard deviation of 
the number of cars he would expect to watch?
\end{parts}
}{}
