


%_______________
\newpage\subsection*{Exercises} % Hypothesis testing

% 1

\eoce{\qt{Identify hypotheses, Part I\label{
}}
Write the null and alternative hypotheses in words and then symbols
for each of the following  situations.
\begin{parts}
\item
    A tutoring company would like to understand if most
    students tend to improve their grades (or not) after
    they use their services.
    They sample 200 of the students who used their service
    in the past year and ask them if their grades have
    improved or declined from the previous year.
\item
    Employers at a firm are worried about the effect of March Madness,
    a basketball championship held each spring in the US, on employee
    productivity.
    They estimate that on a regular business day employees spend on
    average 15 minutes of company time checking personal email,
    making personal phone calls, etc.
    They also collect data on how much company time employees spend
    on such non-business activities during March Madness.
    They want to determine if these data provide convincing evidence
    that employee productivity changed during March Madness.
\end{parts}
}{}

% 2

\eoce{\qt{Identify hypotheses, Part II\label{identify_hypotheses_prop_and_mean_2}} 
Write the null and alternative hypotheses in words and using symbols 
for each of the following situations.
\begin{parts}
\item
    Since 2008, chain restaurants in California have been required
    to display calorie counts of each menu item. Prior to menus
    displaying calorie counts, the average calorie intake of diners
    at a restaurant was 1100 calories.
    After calorie counts started to be displayed on menus,
    a nutritionist collected data on the number of calories consumed
    at this restaurant from a random sample of diners.
    Do these data provide convincing evidence of a difference in the
    average calorie intake of a diners at this restaurant?
\item
    The state of Wisconsin would like to understand
    the fraction of its adult residents that consumed alcohol
    in the last year,
    specifically if the rate is different from the
    national rate of 70\%.
    To help them answer this question, they conduct
    a random sample of 852 residents and ask them
    about their alcohol consumption.
\end{parts}
}{}

% 3

\eoce{\qt{Online communication\label{online_communication_prop_ht_errors}}
A study suggests that 60\% of college student spend
10~or more hours per week communicating with others online.
You believe that this is incorrect and decide to collect your 
own sample for a hypothesis test.
You randomly sample 160 students from your dorm
and find that 70\% spent 10~or more hours a week
communicating with others online.
A~friend of yours, who offers to help you with
the hypothesis test, comes up with the following
set of hypotheses.
Indicate any errors you see.
\begin{align*}
H_0&: \hat{p} < 0.6 \\
H_A&: \hat{p} > 0.7
\end{align*}
}{}

% 4

\eoce{\qt{Married at 25\label{married_at_25_prop_ht_errors}}
A study suggests that the 25\% of 25 year olds have
gotten married.
You believe that this is incorrect and decide to collect
your own sample for a hypothesis test.
From a random sample of 25 year olds in census data
with size 776,
you find that 24\% of them are married.
A friend of yours offers to help you with setting
up hypothesis test and comes up with the following
hypotheses.
Indicate any errors you see.
\begin{align*}
H_0&: \hat{p} = 0.24 \\
H_A&: \hat{p} \neq 0.24
\end{align*}
}{}

% 5

\eoce{\qt{Cyberbullying rates\label{cyberbullying_prop_ci_ht}}
Teens were surveyed about cyberbullying, and
54\% to 64\% reported experiencing cyberbullying
(95\% confidence interval).\footfullcite{pew_cyber_bully_2018}
Answer the following questions based on this interval.
\begin{parts}
\item 
    A newspaper claims that a majority of teens
    have experienced cyberbullying.
    Is this claim supported by the confidence interval?
    Explain your reasoning.
\item\label{cyberbullying_prop_ci_ht_researcher}
    A researcher conjectured that 70\% of teens have
    experienced cyberbullying.
    Is this claim supported by the confidence interval?
    Explain your reasoning.
\item
    Without actually calculating the interval, determine
    if the claim of the researcher from
    part~(\ref{cyberbullying_prop_ci_ht_researcher})
    would be supported based on a 90\% confidence interval?
\end{parts}
}{}

% 6

\eoce{\qt{Waiting at an ER, Part II\label{er_wait_ci_ht_prop_ok}}
Exercise~\ref{er_wait_intro_prop_ok} 
provides a 95\% confidence interval for the mean waiting
time at an emergency room (ER) of (128 minutes, 147 minutes).
Answer the following questions based on this interval.
\begin{parts}
\item
    A local newspaper claims that the average waiting time
    at this ER exceeds 3 hours.
    Is this claim supported by the confidence interval?
    Explain your reasoning.
\item\label{er_wait_ci_ht_prop_ok_dean}
    The Dean of Medicine at this hospital claims the
    average wait time is 2.2 hours.
    Is this claim supported by the confidence interval?
    Explain your reasoning.
\item
    Without actually calculating the interval,
    determine if the claim of the Dean from
    part~(\ref{er_wait_ci_ht_prop_ok_dean})
    would be supported based on a 99\% confidence interval?
\end{parts}
}{}

% 7

\eoce{\qt{Minimum wage, Part 1\label{minimum_wage_prop_1}}
Do a majority of US adults believe raising
the minimum wage will help the economy,
or is there a majority who do not believe this?
A~Rasmussen Reports survey of 1,000 US adults found
that 42\% believe this to be the
case.\footfullcite{webpage:rasmussen-2019-raise-minimum-wage}
Conduct an appropriate hypothesis test to help
answer the research question.
}{}

% 8

\eoce{\qt{Getting enough sleep\label{univ_students_enough_sleep}}
400 students were randomly sampled from a large university,
and 289 said they did not get enough sleep.
Conduct a hypothesis test to check whether this
represents a statistically significant difference
from 50\%, and use a significance level of 0.01.
}{}

% 9

\eoce{\qt{Solving it backwards, Part I\label{backwards_prop_1}}
You are given the following hypotheses:
\begin{align*}
H_0&: p = 0.3 \\
H_A&: p \ne 0.3
\end{align*}
We know the sample size is 90.
For what sample proportion would the p-value be equal to 0.05?
Assume that all conditions  necessary for inference are satisfied.
}{}

% 10

\eoce{\qt{Solving it backwards, Part II\label{backwards_prop_2}}
You are given the following hypotheses:
\begin{align*}
H_0&: p = 0.9 \\
H_A&: p \ne 0.9
\end{align*}
We know that the sample size is 1,429.
For what sample proportion would the p-value be equal to 0.01?
Assume that all conditions necessary for inference are satisfied.
}{}

% 11

\eoce{\qt{Testing for Fibromyalgia\label{errors_fibromyalgia}} A patient named Diana 
was diagnosed with Fibromyalgia, a long-term syndrome of body pain, and was 
prescribed anti-depressants. Being the skeptic that she is, Diana didn't 
initially believe that anti-depressants would help her symptoms. However after 
a couple months of being on the medication she decides that the 
anti-depressants are working, because she feels like her symptoms are in fact 
getting better.
\begin{parts}
\item Write the hypotheses in words for Diana's skeptical position when she 
started taking the anti-depressants.
\item What is a Type~1 Error in this context?
\item What is a Type~2 Error in this context?
\end{parts}
}{}

% 12

\eoce{\qtq{Which is higher\label{prop_which_higher_found_inf}}
In each part below, there is a value of interest and two
scenarios (I and II).
For each part, report if the value of interest is larger
under scenario I, scenario II, or whether the value is
equal under the scenarios.
\begin{parts}
\item
     The standard error of $\hat{p}$ when
     (I)~$n = 125$ or (II)~$n = 500$.
\item
    The margin of error of a confidence interval
    when the confidence level is
    (I)~90\% or (II)~80\%.
\item
    The p-value for a Z-statistic of 2.5 calculated
    based on a (I)~sample with $n = 500$ or based on
    a (II)~sample with $n = 1000$.
\item
    The probability of making a Type~2 Error when the
    alternative hypothesis is true and the significance
    level is (I)~0.05 or (II)~0.10.
\end{parts}
}{}

% 13

\eoce{\qt{Relaxing after work\label{relax_after_work}} The General Social Survey asked the question:
``After an average work day, about how many hours do you have to relax or pursue 
activities that you enjoy?" to a random sample of 1,155 Americans.\footfullcite{data:gss} A 95\% confidence interval for the mean number of hours spent 
relaxing or pursuing activities they enjoy was (1.38, 1.92).
\begin{parts}
\item Interpret this interval in context of the data.
\item Suppose another set of researchers reported a confidence interval with a 
larger margin of error based on the same sample of 1,155 Americans. How does 
their confidence level compare to the confidence level of the interval stated 
above?
\item Suppose next year a new survey asking the same question is conducted, and 
this time the sample size is 2,500. Assuming that the population 
characteristics, with respect to how much time people spend relaxing after work, 
have not changed much within a year. How will the margin of error of the 95\% 
confidence interval constructed based on data from the new survey compare to the 
margin of error of the interval stated above?
\end{parts}
}{}

% 14

\eoce{\qt{Minimum wage, Part 2\label{minimum_wage_prop_2}}
In Exercise~\ref{minimum_wage_prop_1},
we learned that a Rasmussen Reports survey
of 1,000 US adults found that 42\% believe
raising the minimum wage will help the economy.
Construct a 99\% confidence interval for the
true proportion of US adults who believe this.
}{}

% 15

\eoce{\qt{Testing for food safety\label{errors_food_safety}} A food safety inspector 
is called upon to investigate a restaurant with a few customer reports of poor 
sanitation practices. The food safety inspector uses a hypothesis testing 
framework to evaluate whether regulations are not being met. If he decides 
the restaurant is in gross violation, its license to serve food will be revoked.
\begin{parts}
\item Write the hypotheses in words.
\item What is a Type~1 Error in this context?
\item What is a Type~2 Error in this context?
\item Which error is more problematic for the restaurant owner? Why?
\item Which error is more problematic for the diners? Why?
\item As a diner, would you prefer that the food safety inspector requires 
strong evidence or very strong evidence of health concerns before revoking a 
restaurant's license? Explain your reasoning.
\end{parts}
}{}

% 16

\eoce{\qt{True or false\label{tf_found_inf_prop_friendly}}
Determine if the following statements are true or false, and 
explain your reasoning. If false, state how it could be corrected.
\begin{parts}
\item If a given value (for example, the null hypothesized value of a parameter) 
is within a 95\% confidence interval, it will also be within a 99\% confidence 
interval.
\item Decreasing the significance level ($\alpha$) will increase the probability 
of making a Type~1 Error.
\item Suppose the null hypothesis is $p = 0.5$ and we fail to reject $H_0$. 
Under this scenario, the true population proportion is 0.5.
\item With large sample sizes, even small differences between the null value and 
the observed point estimate, a difference often called the
effect size\index{effect size}, will be identified as statistically significant.
\end{parts}
}{}

% 17

\eoce{\qt{Unemployment and relationship problems\label{unemployment_relationship}} 
A USA Today/Gallup poll asked a group of
unemployed and underemployed Americans if they have
had major problems in their  relationships with their
spouse or another close family member as a result of
not having a job (if unemployed) or not having
a full-time job (if underemployed).
27\%~of the 1,145 unemployed respondents and
25\%~of the 675 underemployed respondents said they had
major problems in relationships as a  result of their
employment status.
\begin{parts}
\item
    What are the hypotheses for evaluating if the proportions
    of unemployed and underemployed people who had relationship
    problems were different?
\item
    The p-value for this hypothesis test is approximately 0.35.
    Explain what this means in context of the hypothesis test
    and the data.
\end{parts}
}{}

% 18

\eoce{\qt{Nearsighted\label{nearsighted_updated}}
It is believed that nearsightedness affects about 8\% of 
all children.
In a random sample of 194 children, 21 are nearsighted.
Conduct a hypothesis test for the following question:
do these data provide evidence that the 8\% value is inaccurate?
}{}

% 19

\eoce{\qt{Nutrition labels\label{nutrition_labels}}
The nutrition label on a bag of potato chips says
that a one ounce (28~gram) serving of potato chips
has 130 calories and contains ten grams of fat,
with three grams of saturated fat.
A~random sample of 35 bags yielded
a confidence interval for the number of calories
per bag of 128.2 to 139.8 calories.
Is there evidence that the nutrition label does not 
provide an accurate measure of calories in the bags
of potato chips?
}{}

% 20

\eoce{\qt{CLT for proportions\label{CLT_prop}}
Define the term ``sampling distribution" of the sample proportion,
and describe how the shape, center, and spread of the sampling
distribution change as the sample size increases when $p = 0.1$.
}{}

% 21

\eoce{\qt{Practical vs. statistical\label{prac_stat_sig}} Determine whether the 
following statement is true or false, and explain your reasoning: ``With large 
sample sizes, even small differences between the null value and the observed  
point estimate can be statistically significant."
}{}

% 22

\eoce{\qt{Same observation, different sample size\label{same_obs_diff_n}} Suppose you 
conduct a hypothesis test based on a sample where the sample size is $n = 50$, 
and arrive at a p-value of 0.08. You then refer back to your notes and discover 
that you made a careless mistake, the sample size should have been $n = 500$. 
Will your p-value increase, decrease, or stay the same? Explain.
}{}

% 23

\eoce{\qt{Gender pay gap in medicine\label{gender_pay_gap_medicine}}
A study examined the average pay for men and women
entering the workforce as doctors for 21 different
positions.\footfullcite{LoSassoMedicineGenderPayGap}
\begin{parts}
\item\label{gender_pay_gap_medicine_hypotheses}
    If each gender was equally paid, then we would expect
    about half of those positions to have men paid more
    than women and women would be paid more than men in
    the other half of positions.
    Write appropriate hypotheses to test this scenario.
\item
    Of the 21 positions, men were paid more in 19 of
    those positions, on average.
    Evaluate your hypotheses
    from~(\ref{gender_pay_gap_medicine_hypotheses}).
\end{parts}
}{}
