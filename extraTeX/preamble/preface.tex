\chapter*{{\color{oiB}Predgovor}}
%\chaptertext{}
%\sectiontext{}

\noindent%
Udžbenik Uvod u statistiku pokriva gradivo prvog predmeta iz statistike,
te pruža jasan, koncizan i pristupačan matematički utemeljen uvod u primijenjenu statistiku.
Udžbenik je namijenjen preddiplomskim studentima,
ali se često koristi i u nastavi na srednješkolskoj i diplomskoj razini.
\vspace{3mm}

Nadamo se da će čitaoci, uz to što će usvojiti temelje statističkog razmišljanja i metodologije, prihvatiti tri ideje.\vspace{-1mm}
\begin{itemize}
\setlength{\itemsep}{0mm}
\item
    Statistika je primijenjena znanost sa širokim područjem praktične primjene.
\item
    Ne morate biti matematički guru da biste nešto naučili iz stvarnih, zanimljivih podataka.
\item
    Podaci su neuredni, statistički alati su nesavršeni.
    Ali, kada razumijete snage i slabosti ovih alata, možete ih koristiti da učite o svijetu.
\end{itemize}


%\subsection*{Is this a data science book?}
%
%\noindent%
%Short answer: yes.
%Long answer: it depends what you mean by \term{data science},
%since two types of data scientists have emerged.
%\vspace{3mm}
%
%\noindent%
%Type~A data scientists focus on \emph{analysis},
%such as exploratory data analysis, inference,
%model building, and other related topics.
%Type~B data scientists focus on \emph{building},
%typically in the form of machine learning models
%or other systems.
%As you might expect, these two types share many skills,
%though their main focuses differ.
%This book focuses on skills most commonly used by
%Type~A data scientists.
%For more thoughts, please check out the following page:
%\begin{center}
%\oiRedirect{data_science_types}{{\color{red}BROKEN}}
%\end{center}
%\vspace{3mm}
%
%\noindent%


\subsection*{{\color{oiB}Pregled sadržaja udžbenika}}

\noindent%
Udžbenik sadrži sljedeća poglavlja:%\vspace{2mm}
\begin{description}
\setlength{\itemsep}{0mm}
\item[1. Uvod u podatke.]
    Strukture podataka, varijable
    i osnovne tehnike prikupljanja podataka.
\item[2. Opisivanje podataka.]
    Opisivanje podataka, grafički prikazi (vizualizacija),
    i uvod u zaključivanje koristeći randomizaciju.
\item[3. Vjerojatnost.]
    Osnovna načela teorije vjerojatnosti.
    %This chapter is not required for the later chapters.
\item[4. Razdiobe slučajnih varijabli.]
    Model normalne razdiobe i druge ključne razdiobe.
\item[5. Osnove statističkog zaključivanja.]
    %Introduction to uncertainty in point estimates,
    %confidence intervals, and hypothesis tests.
    Osnovne ideje statističkog zaključivanja u kontekstu
    procjene populacijske proporcije.
\item[6. Zaključivanje o kvalitativnim podacima.]
    Zaključivanje o proporcijama i tablicama primjenom normalne
    i hi-kvadrat razdiobe.
\item[7. Zaključivanje o kvantitativnim podacima.]
    Zaključivanje o aritmetičkim sredinama na jednom ili dva uzorka primjenom
    Studentove \mbox{$t$-razdiobe},
    statistička snaga za usporedbu dvije grupe
    i usporedba više aritmetičkih sredina primjenom analize varijance.
\item[8. Uvod u linearnu regresiju.]
    Regresija za kvantitaivni ishod s jednom nezavisnom varijablom.
    Većinu ovog poglavlja moguće je obraditi nakon 
    poglavlja~\ref{introductionToData}.
\item[9. Multipla i logistička regresija.]
    Regresija za kvantitativne i kvalitativne podatke uz 
    korištenje više nezavisnih varijabli. %for an accelerated course.
\end{description}


%\newpage

\noindent%
\emph{Uvod u statistiku} podržava fleksibilnost 
u izboru i redoslijedu tema.
Ako je glavni cilj doći do multiple regresije
(poglavlje~\ref{ch_regr_mult_and_log})
što je brže moguće, ovo su idealni preduvjeti:
\begin{itemize}
\setlength{\itemsep}{0mm}
\item Poglavlje~\ref{ch_intro_to_data},
    odjeljak~\ref{numericalData},
    i odjeljak~\ref{categoricalData} za solidan
    uvod u strukture podataka i deskriptivne statistike
    koje se koriste u cijelom udžbeniku.
\item Odjeljak~\ref{normalDist}
    za dobro razumijevanje normalne razdiobe.
\item Poglavlje~\ref{ch_foundations_for_inf}
    za razumijevanje osnovnih alata za statističko zaključivanje.
%\item Section~\ref{oneSampleMeansWithTDistribution}
%    and Chapter~\ref{ch_regr_simple_linear}
%    provide required for multiple regression with a numerical
%    outcome.
%    For the remaining chapters, they could be tackled in
%    almost any order, with the exception that
%    
%    
%    come before Chapter~\ref{ch_regr_mult_and_log}.
\item Odjeljak~\ref{oneSampleMeansWithTDistribution}
    za osnove Studentove $t$-razdiobe
\item Poglavlje~\ref{ch_regr_simple_linear}
    za razumijevanje ideja i načela jednostavne regresije
    (s jednim prediktorom).
%    introduce the 
%    which introduces the $t$-distribution, should come before
%    Section~\ref{oneSampleMeansWithTDistribution}
%Chapters~\ref{ch_inference_for_props}-\ref{ch_regr_mult_and_log},
%    could be tackled in
%    almost any order, with the exception that
%    Section~\ref{oneSampleMeansWithTDistribution}
%    and Chapter~\ref{ch_regr_simple_linear}
%    come before Chapter~\ref{ch_regr_mult_and_log}.
%\item Sections~\ref{ch_inference_for_props}
%    and~\ref{} are recommended before logistic regression.
\end{itemize}
%One conspicuously missing topic from the list above is the
%chapter on Probability.
%While useful for a deeper understanding of the calculations,
%especially for anyone looking to take a second course in
%statistics, it is not required reading when the focus is on
%applied data analysis.


\subsection*{{\color{oiB}Primjeri i vježbe}}
%, and appendices}

\noindent%
Primjeri pomažu stjecanju razumijevanja kako primijeniti metode.

\begin{examplewrap}
\begin{nexample}{Ovo je primjer.
  Kada je ovdje postavljeno pitanje, gdje se može naći odgovor?}
  Odgovor je ovdje, u odjeljku s rješenjima primjera!
\end{nexample}
\end{examplewrap}

\noindent%
Kada mislimo da bi čitalac trebao biti spreman pokušati
riješiti primjer, takav primjer navodimo kao Vođenu vježbu.

\begin{exercisewrap}
\begin{nexercise}
Čitalac može provjeriti rješenje ili naučiti kako riješiti problem iz Vođene vježbe
tako da pogleda puno rješenje u fusnoti.\footnotemark{}
%Readers are strongly encouraged to attempt these practice problems.
\end{nexercise}
\end{exercisewrap}
\footnotetext{Svrha Vođenih vježbi je da vas natjeraju na promišljanje, a točnost rješenja
	možete provjeriti u fusnoti Vođene vježbe.}

\noindent%
Vježbe se nalaze i na kraju svakog odjeljka, a vježbe za ponavljanje na kraju svakog poglavlja.
Rješenja neparnih vježbi nalaze se u prilogu~\ref{eoceSolutions}.
%Probability tables for the normal, $t$,
%and chi-square distributions are in
%Appendix~\ref{distributionTables}.


\subsection*{{\color{oiB}Dodatni resursi}}

Video snimke, prezentacije, laboratorijske vježbe sa statističkim sotverom,
skupovi podataka korišteni u udžbeniku i
mnogi drugi resursi (na engleskom jeziku, op.prev.) dostupni su na stranicama\\[-5mm]
\begin{center}
\oiRedirect{os}
    {\color{black}\textbf{openintro.org/os}}
\end{center}
%Data sets for this textbook are available on the website
%and in a companion R package.\footnote{Diez DM,
%    Barr CD, \c{C}etinkaya-Rundel M. 2015.
%    \texttt{openintro}: OpenIntro data sets and supplement
%    functions.
%    \oiRedirect{textbook-github_openintro}
%        {github.com/OpenIntroOrg/openintro-r-package}.}
%All of these resources are free and may be used with
%or without this textbook as a companion.
Pristup podacima korištenim u ovom udžbeniku olakšava i
prilog~\ref{appendix_data},
koji pruža dodatne informacije o svakom skupu podataka
koji se koristi u osnovnom tekstu udžbenika, što je novi sadržaj u četvrtom izdanju.
Za svaki od navedenih skupova podataka postoje i online smjernice (na engleskom jeziku, op.prev.) na stranici
\oiRedirect{data}
    {\color{black}\textbf{openintro.org/data}}
i u okviru pratećeg R paketa
\oiRedirect{textbook-github_openintro}.
% Official:
% http://www.openintro.org/package/openintro
% Currently redirect it to:
% http://openintrostat.github.io/openintro-r-package/
\vspace{3mm}

\subsection*{{\color{oiB}Zahvale}}
Ovaj projekt ne bi bio moguć bez strasti i predanosti 
velikog broja osoba koje nisu navedene kao autori.
Autori žele zahvaliti osoblju
\oiRedirect{textbook-openintro_about}{OpenIntro Staff}
na suradnji i kontinuiranom doprinosu.
Zahvalni smo i stotinama studenata i 
nastavnika koji su nam pružili vrijedne povratne informacije 
od kada smo prvi puta objavili sadržaj udžbenika 2009. godine. \vspace{3mm}

\noindent%
Želimo zahvaliti i mnogim učiteljima koji su pomagali u recenziji ovog izdanja, među kojima su
Laura Acion,
\oiRedirect{matthew_e_aiello-lammens}
    {Matthew E. Aiello-Lammens},
\oiRedirect{jonathan_akin}{Jonathan Akin},
Stacey C. Behrensmeyer,
Juan Gomez,
Jo Hardin,
\oiRedirect{nicholas_horton}{Nicholas Horton},
\oiRedirect{danish_khan}{Danish Khan},
\oiRedirect{peter_hm_klaren}{Peter H.M. Klaren},
Jesse Mostipak,
Jon C. New,
Mario Orsi,
Steve Phelps,
i David Rockoff.
Cijenimo sve njihove komentare, koji su nam pomogli bolje prilagoditi tekst i tako znatno unaprijediti ovaj udžbenik.
