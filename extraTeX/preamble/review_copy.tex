\chapter*{Feedback Instructions}
%\chaptertext{}
%\sectiontext{}

This is a review copy of an unfinished version of the
Fourth Edition of OpenIntro Statistics.
Please read these next few pages before reviewing this book.


\subsection*{What *not* to watch for}

\noindent%
There are several components that you should ignore.
\begin{enumerate}
\setlength{\itemsep}{0mm}
\item
    \textbf{End-of-section/chapter exercises
    and odd-numbered solutions will be included
    in the final version.}
    The newer exercises are not yet ready for sharing,
    so we've omitted exercises from this review copy
    to avoid any confusion.
\item
    \Comment{This is comment text that we are using to
    call out items and that you can consider as FYIs.}
    There's a big dot in the margin that makes it easy
    to spot these notes.
\item
    There are plenty of formatting issues,
    e.g. awkward page breaks or footnotes on the wrong page.
    These issues will be fixed during final textbook formatting.
\item
    There are some broken references such as
    ``Figure~\ref{}'' or ``Section~\ref{}''.
    Any such references will be fixed before
    the Fourth Edition is released.
\end{enumerate}


\subsection*{We will send a survey for you to complete}

\noindent%
We will send you a survey by December 31st.
Responding to this survey by January 7th will be most
helpful to us, which is when we will be starting to
incorporate significant amounts of feedback.


\subsection*{Sending feedback as you read}

\noindent%
If you are browsing through the book and think,
``Hey, they should add / do / change / etc [thing]'',
send a note to \url{os4@openintro.org}
or via
\href{http://www.openintro.org/os4}{\texttt{openintro.org/os4}}
\vspace{3mm}

\noindent%
Below are specific topics where you may
want to voice your thoughts:
\begin{enumerate}
\item
    If you are reading an example
    or case study and think that there's an
    interesting comment that might be made on
    confounding variables or on what a multivariate
    analysis would be like, please let us know.
    We'll be adding such comments and discussion
    during January and February.
\item
    If you read the new
    \emph{Foundations for Inference} chapter,
    what do you think about it?
    Do you like, dislike, or not care that we
    now introduce inference using proportions
    before means?
\item
    We have also reversed the ordering of the two chapters
    covering inference for proportions / means.
    Do we move too quickly or too slowly in spots
    for either section?
    Which spots require more explanation or examples?
\item
    The new case study for logistic regression
    covers a sensitive yet important topic:
    racial discrimination.
    If you read this section, do you think the
    topic was presented and discussed in an
    appropriately respectful and responsible way?

    We will also be getting a thorough
    review by subject-matter experts for this section.
%\item
%    In newer examples, we more strongly suggest software
%    over using tables for finding tail areas.
%    We are planning to do further changes around wording
%    in existing examples and would like feedback on this
%    direction.
%\item
%    The 3rd Edition launched with only black-and-white
%    paperbacks, and a year after launch we made
%    full color hardcovers available.
%    How important is it to you that we offer
%    (1) full-color books available and/or
%    (2) hardcover textbooks available?
%    (Our tentative plan is to launch with
%    a black-and-white paperback and also
%    a full-color paperback, where the expected
%    prices are \$20 and \$35, respectively.)
%\item

\end{enumerate}


\subsection*{Some of the changes already implemented}

\noindent%
The following sections contained notable updates
in content or examples:
\begin{itemize}
%\setlength{\itemsep}{0mm}
\item 1.2,
\item 1.3.4,
\item all of Chapter~\ref{ch_summarizing_data},
\item some loan data examples in 3.1,
\item stock return examples in 3.4,
\item (nothing notable in Chapter~\ref{ch_distributions}]),
\item all of Chapter~\ref{ch_foundations_for_inf},
\item 6.1.2,
\item 6.1.3,
\item 6.3.5,
\item 6.4,
\item 7.1.5,
\item 7.2,
\item 7.5 (updated MLB data),
\item 8.4 (updated election data),
\item 9.4
\end{itemize}


\noindent%
Here are some special callouts for changes made:
\begin{description}
\item[Stylistic.]
    Each section now starts at the top of a page.
    Section, subsection, term boxes, tip boxes,
    examples, and guided practice
    all have updated appearances.

    There are some bugs with spacing here and there,
    e.g. with sections and the horizontal lines,
    that we are still working out.

    Video and slide icons / links have also been removed,
    since these will be presented in a different way
    in the Fourth Edition.
\item[Graphics and statistical summaries get their own chapter.]
    The first chapter of the Third Edition has been
    broken into two chapters in the Fourth Edition.
\item[Inference: proportions before means.]
    We introduce inference using proportion before means
    in the Fourth Edition.
    The inference of proportions chapter also now
    precedes the inference for means chapter.
\item[Simulation and randomization.]
    Two sections in the inference for proportions
    in small sample situations have been removed
    and will become online extras in about April.
    The randomization case study section near the start of the
    textbook was retained with a new case study.
\item[Lots of new examples.]
    We have replaced or updated many older or less interesting
    data sets with new case studies to make the book more
    engaging for both students and teachers.
    (A few lingering instances remain that will be resolved
    before the Fourth Edition is complete.)
    If any data sets strike you as outdated or uninteresting,
    please send a note.
\item

\end{description}


\subsection*{Changes in progress or that will be completed}

\noindent%
For reference, we will go to print in April.

\noindent%
Below are tentative changes, and we welcome
feedback and suggestions on these plans.
\begin{enumerate}
%\item
%    As earlier mentioned, exercises will be moved to the
%    end of sections, and there will be some new exercises
%    in the new edition.
\item
    \textbf{We are moving all data references into an
    appendix and out of footnotes in the text}
    (you can observe in the book that many footnotes
    for references have disappeared).
    Our goals with this change are to
    \begin{enumerate}
    \item
        simplify reading for the large majority of readers, and
    \item
        provide a place where we can provide a complete
        list of all data sets in the text.
    \end{enumerate}
    The appendix will also include links (in the PDF)
    to pages dedicated to each data set
    and a CSV download link.
\item
    \textbf{We are tentatively planning to place exercises
    at the end of each section.}
    We would also include a handful of exercises at the
    end of each chapter that would be more comprehensive.
\item
    Create a couple lead-in pages for each chapter that
    stand out more strongly.
    Designs have been drawn up but are not yet implemented
    in the \LaTeX{} source files.
\item
    Replace the Mario Kart auction data in
    Chapter~9 with a new data set that is to-be-determined.
\item
    We are cutting out the condition that the
    \emph{sample size needs to be $\leq 10\%$ of the
    population size}.
    It will be mentioned briefly as a consideration
    but no longer included as a condition.
%    We've received several cases of feedback that this
%    is confusing (often asked: why is collecting more data bad?),
%    or that it is not practically relevant except
%    in very rare cases.
%    If you are concerned about this change,
%    please let us know.
\item
    The discussion of statistical vs practical significance
    is not in the new \emph{Foundations for Inference} chapter.
    However, it will be added back into the book before the
    Fourth Edition is released in a location to-be-determined.
\item
    We will be completing a thorough review of the inference
    chapters to ensure they read well in their new order.
    Most especially, we want to be confident the 2-prop
    description is reasonable since it is no longer preceded
    by the 2-mean scenario.
\item
    We may add a new section on graphics
    that would follow the sections on summarizing numerical
    and categorical data.
\item
    We may include some basics on R code at the end
    of some sections.
    If this is of particular interest to you,
    please let us know.
\item
    We may include some blank pages in the Fourth Edition
    launch if we plan to add specific types of new content.
    This strategy would allow us to add extra (non-critical)
    content later without affecting page numbering of
    textbooks already purchased or downloaded.
\item
    You'll also find several comments throughout the book
    that callout additional items.
\end{enumerate}







