\exercisesheader{}

% 1

\eoce{\qt{Migrena i akupunktura,
    DIO I\label{migraine_and_acupuncture_intro}}
Migrena je posebno bolni oblik glavobolje protiv koje 
pacijenti ponekad žele koristiti akupunkturu.
Kako bi ustanovili da li akupunktira olakšava bolove kod migrene,
istraživači su proveli kontrolirano randomizirano istraživanje
u okviru kojeg je 89 žena s dijagnozom migrene slučajno raspoređeno 
u dvije grupe: izloženu i kontrolnu.
43 pacijentice u izloženoj grupi primile su terapiju 
akupunkturom posebno osmišljenom za liječenje migrene.
46 pacijentica u kontrolnoj grupi primile su 
placebo terapiju (ubod iglom na mjestu koje nije akupunkturno).
24 sata nakon što su primile terapiju, istraživači su pitali pacijentice 
da li su bile bez bolova.
Rezultati su sažeti u donjoj kontingencijskoj tablici.\footfullcite{Allais:2011}

\noindent\begin{minipage}[l]{0.4\textwidth}
\begin{tabular}{ll  cc c} 
			                         		&           & \multicolumn{2}{c}{\textit{Bez bolova}} \\
\cline{3-4}
			                        	 	&			& Da 	& Ne 	                  & Ukupno \\
\cline{2-5}
							& Izložena 	& 10	 	& 33		                  & 43 \\
\raisebox{1.5ex}[0pt]{\emph{Grupa}} & Kontrolna	 	& 2	 	& 44 	 	                  & 46 \\
\cline{2-5}
							& Ukupno		& 12		& 77		                  & 89
\end{tabular}
\end{minipage}
\begin{minipage}[c]{0.05\textwidth}
\end{minipage}
\begin{minipage}[c]{0.27\textwidth}
\begin{center}
\includegraphics[width = 0.75\textwidth]{ch_intro_to_data/figures/eoce/migraine_and_acupuncture_intro/earacupuncture.pdf}
\end{center}
\end{minipage}
\begin{minipage}[c]{0.25\textwidth}
{\footnotesize Slika iz originalnoga znanstvenoga rada koja prikazuje prikladno područje
(M) u odnosu na neprikladno područje (S) koje je korišteno u terapiji protiv migrene.}
\end{minipage}
\begin{parts}
\item Koji postotak pacijentica u izloženoj grupi je bio bez bolova 24 sata nakon primanja terapije akupunkturom? 
\item Koji postotak pacijentica u kontrolnoj grupi je bio bez bolova?
\item U kojoj grupi je postotak pacijentica bez bolova 24 sata nakon terapije bio veći?
\item Vaši dosadašnji rezultati mogli bi sugerirati da je akupunktura učinkovita terapija
protiv migrene za sve osobe koje pate od migrene. Međutim, to nije jedini mogući zaključak
koji se može izvesti iz vaših rezultata. Koje je drugo moguće objašnjenje uočene razlike između grupa u postotku pacijentica koje nisu imale bolove 24 sata nakon terapije akupunkturom?
\end{parts}
}{}

% 2

\eoce{\qt{Upala sinusa i antibiotici,
    Dio I\label{sinusitis_and_antibiotics_intro}} 
Istraživači koji proučavaju učinak terapije antibioticima na akutnu upalu sinusa u usporedbi sa simptomatskim liječenjem slučajno su rasporedili 166 odraslih osoba s dijagnozom akutne upale sinusa u dvije grupe: izloženu i kontrolnu. Sudionici istraživanja dobili su ili 10-dnevnu terapiju amoxicilinom (antibiotik) ili placebom sličnog oblika i okusa. Placebo je sadržavao simptomatsku terapiju kao što je panadon, sredstvo za otčepljivanje nosa i sl. Na kraju 10-dnevnog perioda pacijente su pitali da li su osjetili poboljšanje simptoma. Razdioba odgovora prikazana je u sljedećoj tablici.
\footfullcite{Garbutt:2012}
\begin{center}
\begin{tabular}{ll  cc c} 
                                    			&			& \multicolumn{2}{c}{\textit{Poboljšanje simptoma}} \\
                                    			&			& \multicolumn{2}{c}{\textit{(percepcija pacijenta)}} \\
\cline{3-4}
			                        		&			& Da 	& Ne 	& Ukupno \\
\cline{2-5}
							& Izložena 	& 66		& 19		& 85 \\
\raisebox{1.5ex}[0pt]{\emph{Grupa}}	& Kontrolna		& 65		& 16 		& 81 \\
\cline{2-5}
							& Ukupno		& 131	& 35		& 166
\end{tabular}
\end{center}
\begin{parts}
\item Koji je postotak pacijenata u izloženoj grupi osjetio poboljšanje simptoma? 
\item Koji postotak pacijenata u kontrolnoj grupi je osjetio poboljšanje simptoma?
\item U kojoj grupi je postotak pacijenata koji su osjetili poboljšanje simptoma veći?
\item
    Vaši dosadašnji rezultati mogu upućivati da postoji stvarna razlika u učinkovitosti antibiotika i placeba u poboljšanju simptoma upale sinusa.
    Međutim, to nije jedini mogući zaključak koji se može izvesti iz tih rezultata. Koje je drugo moguće objašnjenje za uočenu razliku u postotku pacijenata koji su osjetili poboljšanje simptoma između izložene i kontrolne grupe?
\end{parts}
}{}
