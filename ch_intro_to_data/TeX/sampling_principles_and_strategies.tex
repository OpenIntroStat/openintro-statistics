\exercisesheader{}

% 13

\eoce{\qt{Air pollution and birth outcomes, scope of inference\label{scope_airpoll}} 
Exercise~\ref{study_components_airpoll} introduces a study where researchers 
collected data to examine the relationship between air pollutants and preterm 
births in Southern California. During the study air pollution levels were 
measured by air quality monitoring stations. Length of gestation data were 
collected on 143,196 births between the years 1989 and 1993, and air pollution 
exposure during gestation was calculated for each birth.
\begin{parts}
\item Identify the population of interest and the sample in this study.
\item Comment on whether or not the results of the study can be generalized to the 
population, and if the findings of the study can be used to establish causal relationships.
\end{parts}
}{}

% 14

\eoce{\qt{Cheaters, scope of inference\label{scope_cheaters}} 
Exercise~\ref{study_components_cheaters} introduces a study where researchers 
studying the relationship between honesty, age, and self-control conducted an 
experiment on 160 children between the ages of 5 and 15. The researchers asked 
each child to toss a fair coin in private and to record the outcome (white or black) 
on a paper sheet, and said they would only reward children who report white. 
Half the students were explicitly told not to cheat and the others were not given 
any explicit instructions. Differences were observed in the cheating rates in the
instruction and no instruction groups, as well as some differences across 
children's characteristics within each group.
\begin{parts}
\item Identify the population of interest and the sample in this study.
\item Comment on whether or not the results of the study can be generalized to the 
population, and if the findings of the study can be used to establish causal 
relationships.
\end{parts}
}{}

% 15

\eoce{\qt{Buteyko method, scope of inference\label{scope_buteyko}} 
Exercise~\ref{study_components_buteyko} introduces a study on using the Buteyko 
shallow breathing technique to reduce asthma symptoms and improve quality of life.
As part of this study 600 asthma patients aged 18-69 who relied on medication for 
asthma treatment were recruited and randomly assigned to two groups: one practiced 
the Buteyko method and the other did not. Those in the Buteyko group experienced,
on average, a significant reduction in asthma symptoms and an improvement in quality 
of life.
\begin{parts}
\item Identify the population of interest and the sample in this study.
\item Comment on whether or not the results of the study can be generalized to the 
population, and if the findings of the study can be used to establish causal 
relationships.
\end{parts}
}{}

% 16

\eoce{\qt{Stealers, scope of inference\label{scope_stealers}} 
Exercise~\ref{study_components_stealers} introduces a study on the relationship 
between socio-economic class and unethical behavior. As part of this study 129 
University of California Berkeley undergraduates were asked to identify themselves 
as having low or high social-class by comparing themselves to others with the most 
(least) money, most (least) education, and most (least) respected jobs. They were 
also presented  with a jar of individually wrapped candies and informed that the
candies were for children in a nearby laboratory, but that they could take some if 
they wanted. After completing some unrelated tasks, participants reported the 
number of candies they had taken. It was found that those who were identified as 
upper-class took more candy than others.
\begin{parts}
\item Identify the population of interest and the sample in this study.
\item Comment on whether or not the results of the study can be generalized to the 
population, and if the findings of the study can be used to establish causal 
relationships.
\end{parts}
}{}

% 17

\eoce{\qt{Relaxing after work\label{relax_after_work_definitions}} The General 
Social Survey asked the question, ``After an average work day, about how many 
hours do you have to relax or pursue activities that you enjoy?" to a random 
sample of 1,155 Americans. The average relaxing time was found to be 1.65 
hours. Determine which of the following is an observation, a variable, a 
sample statistic (value calculated based on the observed sample), or a 
population parameter.
\begin{parts}
\item An American in the sample.
\item Number of hours spent relaxing after an average work day.
\item 1.65.
\item Average number of hours all Americans spend relaxing after an average 
work day.
\end{parts}
}{}

\D{\newpage}

% 18

\eoce{\qt{Cats on YouTube\label{cats_on_youtube_definitions}} Suppose you want to 
estimate the percentage of videos on YouTube that are cat videos. It is 
impossible for you to watch all videos on YouTube so you use a random video 
picker to select 1000 videos for you. You find that 2\% of these videos are 
cat videos.
Determine which of the following is an observation, a variable, 
a sample statistic (value calculated based on the observed sample), 
or a population parameter.
\begin{parts}
\item Percentage of all videos on YouTube that are cat videos.
\item 2\%.
\item A video in your sample.
\item Whether or not a video is a cat video.
\end{parts}
}{}

% 19

\eoce{\qt{Course satisfaction across sections\label{course_satisfaction_sections}} 
A large college class has 160 students. All 160 students attend the lectures 
together, but the students are divided into 4 groups, each of 40 students, 
for lab sections administered by different teaching assistants. The professor 
wants to conduct a survey about how satisfied the students are with the course, 
and he believes that the lab section a student is in might affect the student's 
overall satisfaction with the course.
\begin{parts}
\item What type of study is this?
\item Suggest a sampling strategy for carrying out this study.
\end{parts}
}{}

% 20

\eoce{\qt{Housing proposal across dorms\label{housing_proposal_dorms}} On a large 
college campus first-year students and sophomores live in dorms located on 
the eastern part of the campus and juniors and seniors live in dorms located 
on the western part of the campus. Suppose you want to collect student opinions 
on a new housing structure the college administration is proposing and you want 
to make sure your survey equally represents opinions from students from all years.
\begin{parts}
\item What type of study is this?
\item Suggest a sampling strategy for carrying out this study.
\end{parts}
}{}

% 21

\eoce{\qt{Internet use and life expectancy\label{internet_life_expectancy}} The 
following scatterplot was created as part of a study evaluating the 
relationship between estimated life expectancy at birth (as of 2014) and 
percentage of internet users (as of 2009) in 208 countries for which such 
data were available.\footfullcite{data:ciaFactbook}

\noindent\begin{minipage}[c]{0.44\textwidth}
\begin{parts}
\item Describe the relationship between life expectancy and percentage of 
internet users.
\item What type of study is this?
\item State a possible confounding variable that might explain this relationship 
and describe its potential effect.
\end{parts} \vspace{15mm}
\end{minipage}
\begin{minipage}[r]{0.55\textwidth}
\hfill%
\Figures[Scatterplot with "percent of internet users" (0\% to 100\%) along the horizontal axis and "life expectancy at birth" (50 to 90) along the vertical axis. For 0\% to 15\%, about 100 points are evenly spread between 50 and 75. Then for 15\% to 90\%, the points are concentrated between about 70 and 85, and a slight upward trend is evident.]{0.87}{eoce/internet_life_expectancy}{internet_life_expectancy}
\end{minipage}
}{}

% 22

\eoce{\qt{Stressed out, Part I\label{stressed_out_observational}} A study that 
surveyed a random sample of otherwise healthy high school students found that 
they are more likely to get muscle cramps when they are stressed. The study 
also noted that students drink more coffee and sleep less when they are 
stressed.
\begin{parts}
\item What type of study is this?
\item Can this study be used to conclude a causal relationship between 
increased stress and muscle cramps?
\item State possible confounding variables that might explain the observed 
relationship between increased stress and muscle cramps. 
\end{parts}
}{}

% 23

\eoce{\qt{Evaluate sampling methods\label{evaluate_sampling_methods}} A university wants to 
determine what fraction of its undergraduate student body support a new \$25 annual fee 
to improve the student union. For each proposed method below, indicate whether 
the method is reasonable or not.
\begin{parts}
\item Survey a simple random sample of 500 students.
\item Stratify students by their field of study, then sample 10\% of students from  
each stratum.
\item Cluster students by their ages (e.g. 18 years old in one cluster, 19 years 
old in one cluster, etc.), then randomly sample three clusters and survey all 
students in those clusters.
\end{parts}
}{}

\D{\newpage}

% 24

\eoce{\qt{Random digit dialing\label{random_digit_dialing}} The Gallup Poll uses a 
procedure called random digit dialing, which creates phone numbers based on 
a list of all area codes in America in conjunction with the associated number 
of residential households in each area code. Give a possible reason the Gallup 
Poll chooses to use random digit dialing instead of picking phone numbers 
from the phone book.
}{}

% 25

\eoce{\qt{Haters are gonna hate, study confirms\label{scope_haters}}
A study published in the
\textit{Journal of Personality and Social Psychology}
asked a group of 200 randomly sampled men and
women to evaluate how they felt about various subjects,
such as camping, health care, architecture, taxidermy, 
crossword puzzles, and Japan in order to measure their
attitude towards mostly independent stimuli.
Then, they presented the participants with information
about a new product: a microwave oven. This microwave oven
does not exist, but the participants didn't know this,
and were given three positive and three negative fake reviews.
People who reacted positively to the subjects on the
dispositional attitude measurement also tended to react 
positively to the microwave oven, and those who reacted
negatively tended to react negatively to it.
Researchers concluded that ``some people tend to 
like things, whereas others tend to dislike things, and a more thorough 
understanding of this tendency will lead to a more thorough understanding of 
the psychology of attitudes." \footfullcite{Hepler:2013}
\begin{parts}
\item What are the cases?
\item What is (are) the response variable(s) in this study?
\item What is (are) the explanatory variable(s) in this study?
\item Does the study employ random sampling?
\item Is this an observational study or an experiment? Explain your reasoning.
\item Can we establish a causal link between the explanatory and response 
variables?
\item Can the results of the study be generalized to the population at large?
\end{parts}
}{}

% 26

\eoce{\qt{Family size\label{family_size}} Suppose we want to estimate household 
size, where a ``household" is defined as people living together in the 
same dwelling, and sharing living accommodations. If we select students 
at random at an elementary school and ask them what their family size is, 
will this be a good measure of household size? Or will our average be 
biased? If so, will it overestimate or underestimate the true value?
}{}

% 27

\eoce{\qt{Sampling strategies\label{sampling_strategies}} A statistics student who is curious about the relationship between the amount of time students spend on social networking sites and their performance at school decides to conduct a survey. Various research strategies for collecting data are described below. In each, name the sampling method proposed and any bias you might expect.
\begin{parts}
\item He randomly samples 40 students from the study's population, gives them the survey, asks them to fill it out and bring it back the next day.
\item He gives out the survey only to his friends, making sure each one of them fills out the survey.
\item He posts a link to an online survey on Facebook and asks his friends to fill out the survey.
\item He randomly samples 5 classes and asks a random sample of students from those classes to fill out the survey.
\end{parts}
}{}

% 28

\eoce{\qt{Reading the paper\label{reading_paper}} Below are excerpts from two 
articles published in the \emph{NY Times}:
\begin{parts}
\item An article titled \emph{Risks: Smokers Found More Prone to Dementia} 
states the following: \footfullcite{news:smokingDementia}
\begin{adjustwidth}{1em}{1em}
{\footnotesize ``Researchers analyzed data from 23,123 health plan members who 
participated in a voluntary exam and health behavior survey from 1978 to 1985, 
when they were 50-60 years old. 23 years later, about 25\% of the group had 
dementia, including 1,136 with Alzheimer's disease and 416 with vascular 
dementia. After adjusting for other factors, the researchers concluded that 
pack-a-day smokers were 37\% more likely than nonsmokers to develop dementia, 
and the risks went up with increased smoking; 44\% for one to two packs a day; 
and twice the risk for more than two packs."}
\end{adjustwidth}
Based on this study, can we conclude that smoking causes dementia later in 
life? Explain your reasoning.
\item Another article titled \emph{The School Bully Is Sleepy} states the 
following: \footfullcite{news:bullySleep}
\begin{adjustwidth}{1em}{1em}
{\footnotesize ``The University of Michigan study, collected survey data from 
parents on each child's sleep habits and asked both parents and teachers to 
assess behavioral concerns. About a third of the students studied were 
identified by parents or teachers as having problems with disruptive behavior 
or bullying. The researchers found that children who had behavioral issues and 
those who were identified as bullies were twice as likely to have shown 
symptoms of sleep disorders."}
\end{adjustwidth}
A friend of yours who read the article says, ``The study shows that sleep 
disorders lead to bullying in school children." Is this statement justified? 
If not, how best can you describe the conclusion that can be drawn from this 
study?
\end{parts}
}{}
