


%_______________
\newpage\subsection*{Exercises} % Inference for linear regression

% 1

\eoce{\qt{Body measurements, Part IV\label{body_measurements_weight_height_inf}} 
The scatterplot and least squares summary below show the relationship 
between weight measured in kilograms and height measured in centimeters 
of 507 physically active individuals.

\noindent\begin{minipage}[c]{0.4\textwidth}
\begin{center}
\includegraphics[width=\textwidth]{ch_regr_simple_linear/figures/eoce/body_measurements_weight_height_inf/body_measurements_weight_height.pdf}
\end{center}
\end{minipage}
\begin{minipage}[c]{0.6\textwidth}
{\scriptsize
\begin{center}
\begin{tabular}{rrrrr}
    \hline
            & Estimate  & Std. Error    & t value   & Pr($>$$|$t$|$) \\ 
    \hline
(Intercept) & -105.0113 & 7.5394        & -13.93    & 0.0000 \\ 
height      & 1.0176    & 0.0440        & 23.13     & 0.0000 \\
    \hline
\end{tabular}
\end{center}
}
\end{minipage}
\begin{parts}
\item Describe the relationship between height and weight.
\item Write the equation of the regression line. Interpret the slope 
and intercept in context.
\item Do the data provide strong evidence that an increase in height 
is associated with an increase in weight? State the null and alternative 
hypotheses, report the p-value, and state your conclusion.
\item The correlation coefficient for height and weight is 0.72. 
Calculate $R^2$ and interpret it in context.
\end{parts}
}{}

% 2

\eoce{\qt{Beer and blood alcohol content\label{beer_blood_alcohol_inf}} 
Many people believe that gender, 
weight, drinking habits, and many other factors are much more important 
in predicting blood alcohol content (BAC) than simply considering the 
number of drinks a person consumed. Here we examine data from sixteen 
student volunteers at Ohio State University who each drank a randomly 
assigned number of cans of beer. These students were evenly divided 
between men and women, and they differed in weight and drinking habits. 
Thirty minutes later, a police officer measured their blood alcohol 
content (BAC) in grams of alcohol per deciliter of blood.
\footfullcite{Malkevitc+Lesser:2008} The scatterplot and regression 
table summarize the findings.

\noindent\begin{minipage}[c]{0.4\textwidth}
\begin{center}
\includegraphics[width=\textwidth]{ch_regr_simple_linear/figures/eoce/beer_blood_alcohol_inf/beer_blood_alcohol.pdf}
\end{center}
\end{minipage}
\begin{minipage}[c]{0.6\textwidth}
{\scriptsize
\begin{center}
\begin{tabular}{rrrrr}
    \hline
            & Estimate  & Std. Error    & t value   & Pr($>$$|$t$|$) \\ 
    \hline
(Intercept) & -0.0127   & 0.0126        & -1.00     & 0.3320 \\ 
beers       & 0.0180    & 0.0024        & 7.48      & 0.0000 \\ 
    \hline
\end{tabular}
\end{center}
}
\end{minipage}
\begin{parts}
\item Describe the relationship between the number of cans of beer 
and BAC.
\item Write the equation of the regression line. Interpret the slope 
and intercept in context.
\item Do the data provide strong evidence that drinking more cans of 
beer is associated with an increase in blood alcohol? State the null 
and alternative hypotheses, report the p-value, and state your 
conclusion.
\item The correlation coefficient for number of cans of beer and BAC 
is 0.89. Calculate $R^2$ and interpret it in context.
\item Suppose we visit a bar, ask people how many drinks they have had, 
and also take their BAC. Do you think the relationship between number 
of drinks and BAC would be as strong as the relationship found in the 
Ohio State study?
\end{parts}
}{}

% 3

\eoce{\qt{Husbands and wives, Part II\label{husbands_wives_height_inf}} The 
scatterplot below summarizes husbands' and wives' heights in a random 
sample of 170 married couples in Britain, where both partners' ages are 
below 65 years. Summary output of the least squares fit for predicting 
wife's height from husband's height is also provided in the table.

\noindent\begin{minipage}[c]{0.4\textwidth}
\begin{center}
\includegraphics[width=\textwidth]{ch_regr_simple_linear/figures/eoce/husbands_wives_height_inf/husbands_wives_height.pdf}
\end{center}
\end{minipage}
\begin{minipage}[c]{0.6\textwidth}
{\scriptsize
\begin{center}
\begin{tabular}{rrrrr}
    \hline
                    & Estimate  & Std. Error    & t value   & Pr($>$$|$t$|$) \\ 
    \hline
(Intercept)         & 43.5755   & 4.6842        & 9.30      & 0.0000 \\ 
height\_\hspace{0.3mm}husband   & 0.2863    & 0.0686        & 4.17      & 0.0000 \\ 
    \hline
\end{tabular}
\end{center}
}
\end{minipage}
\begin{parts}
\item Is there strong evidence that taller men marry taller women? 
State the hypotheses and include any information used to conduct the test.
\item Write the equation of the regression line for predicting wife's 
height from husband's height.
\item Interpret the slope and intercept in the context of the application.
\item Given that $R^2 = 0.09$, what is the correlation of heights 
in this data set?
\item You meet a married man from Britain who is 5'9" (69 inches). 
What would you predict his wife's height to be? How reliable is this 
prediction?
\item You meet another married man from Britain who is 6'7" (79 inches). 
Would it be wise to use the same linear model to predict his wife's 
height? Why or why not?
\end{parts}
}{}

% 4

\eoce{\qt{Husbands and wives, Part III\label{husbands_wives_age_inf}}
Exercise~\ref{husbands_wives_height_inf} presents a scatterplot displaying the 
relationship between husbands' and wives' ages in a random sample of 
170 married couples in Britain, where both partners' ages are below 65 
years. Given below is summary output of the least squares fit for 
predicting wife's age from husband's age.

\noindent\begin{minipage}[c]{0.4\textwidth}
\begin{center}
\includegraphics[width=\textwidth]{ch_regr_simple_linear/figures/eoce/husbands_wives_age_inf/husbands_wives_age.pdf}
\end{center}
\end{minipage}
\begin{minipage}[c]{0.6\textwidth}
{\scriptsize
\begin{center}
\begin{tabular}{rrrrr}
  \hline
                & Estimate  & Std. Error    & t value   & Pr($>$$|$t$|$) \\ 
  \hline
(Intercept)     & 1.5740    & 1.1501        & 1.37      & 0.1730 \\ 
age\_\hspace{0.3mm}husband  & 0.9112    & 0.0259        & 35.25     & 0.0000 \\ 
   \hline
\multicolumn{5}{r}{$df = 168$} \\
\end{tabular}
\end{center}
}
\end{minipage}
\begin{parts}
\item We might wonder, is the age difference between husbands and 
wives consistent across ages? If this were the case, then the slope 
parameter would be $\beta_1 = 1$. Use the information above to evaluate 
if there is strong evidence that the difference in husband and wife ages 
differs for different ages.
\item Write the equation of the regression line for predicting wife's 
age from husband's age.
\item Interpret the slope and intercept in context.
\item Given that $R^2 = 0.88$, what is the correlation of ages  in 
this data set?
\item You meet a married man from Britain who is 55 years old. What 
would you predict his wife's age to be? How reliable is this prediction?
\item You meet another married man from Britain who is 85 years old. 
Would it be wise to use the same linear model to predict his wife's 
age? Explain.
\end{parts}
}{}

% 5

\eoce{$\:$ \\
\noindent \begin{minipage}[c]{0.56\textwidth}
\qt{Urban homeowners, Part II\label{urban_homeowners_cond}}
Exercise~\ref{urban_homeowners_outlier} gives a scatterplot displaying the 
relationship between the percent of families that own their home and 
the percent of the population living in urban areas. Below is a 
similar scatterplot, excluding District of Columbia, as well as the 
residuals plot. There were 51 cases.
\begin{parts}
\item For these data, $R^2=0.28$. What is the correlation? How can 
you tell if it is positive or negative?
\item Examine the residual plot. What do you observe? Is a simple 
least squares fit appropriate for these data?
\end{parts}
\vspace{15mm}
\end{minipage}
\begin{minipage}[c]{0.02\textwidth}
$\:$ \\
\end{minipage}
\begin{minipage}[c]{0.4\textwidth}
\begin{center}
\includegraphics[width=\textwidth]{ch_regr_simple_linear/figures/eoce/urban_homeowners_cond/urban_homeowners_cond.pdf}
\end{center}
\end{minipage}
}{}

% 6

\eoce{\qt{Rate my professor\label{rate_my_prof}} Many college courses conclude by giving 
students the opportunity to evaluate the course and the instructor 
anonymously. However, the use of these student evaluations as an 
indicator of course quality and teaching effectiveness is often 
criticized because these measures may reflect the influence of non-
teaching related characteristics, such as the physical appearance of 
the instructor. Researchers at University of Texas, Austin collected 
data on teaching evaluation score (higher score means better) and 
standardized beauty score (a score of 0 means average, negative score 
means below average, and a positive score means above average) for a 
sample of 463 professors.\footfullcite{Hamermesh:2005} The 
scatterplot below shows the relationship between these variables, and 
also provided is a regression output for predicting teaching 
evaluation score from beauty score.

\noindent\begin{minipage}[c]{0.4\textwidth}
\includegraphics[width=\textwidth]{ch_regr_simple_linear/figures/eoce/rate_my_prof/rate_my_prof_eval_beauty.pdf} \\
\end{minipage}
\begin{minipage}[c]{0.6\textwidth}
\begin{tabular}{rrrrr}
    \hline
            & Estimate  & Std. Error    & t value   & Pr($>$$|$t$|$) \\ 
  \hline
(Intercept) & 4.010     & 0.0255        & 	157.21  & 0.0000 \\ 
beauty      &  \fbox{\textcolor{white}{{\footnotesize Cell 1}}}  
                        & 0.0322        & 4.13      & 0.0000\vspace{0.8mm} \\ 
   \hline
\end{tabular}
\end{minipage}
\begin{parts}
\item Given that the average standardized beauty score is -0.0883 and 
average teaching evaluation score is 3.9983, calculate the slope. 
Alternatively, the slope may be computed using just the information 
provided in the model summary table.
\item Do these data provide convincing evidence that the slope of the 
relationship between teaching evaluation and beauty is positive? 
Explain your reasoning.
\item List the conditions required for linear regression and check if 
each one is satisfied for this model based on the following 
diagnostic plots.
\begin{center}
\includegraphics[width=0.4\textwidth]{ch_regr_simple_linear/figures/eoce/rate_my_prof/rate_my_prof_residuals.pdf}
\includegraphics[width=0.4\textwidth]{ch_regr_simple_linear/figures/eoce/rate_my_prof/rate_my_prof_residuals_hist.pdf} \\
\includegraphics[width=0.4\textwidth]{ch_regr_simple_linear/figures/eoce/rate_my_prof/rate_my_prof_residuals_qq.pdf}
\includegraphics[width=0.4\textwidth]{ch_regr_simple_linear/figures/eoce/rate_my_prof/rate_my_prof_residuals_order.pdf}
\end{center}
\end{parts}
}{}

% 7

\eoce{\qt{Murders and poverty, Part II\label{murders_poverty_inf}}
Exercise~\ref{murders_poverty_reg} presents regression output from a model 
for predicting annual murders per million from percentage living in 
poverty based on a random sample of 20 metropolitan areas. The model 
output is also provided below.
\begin{center}
\begin{tabular}{rrrrr}
    \hline
            & Estimate  & Std. Error    & t value   & Pr($>$$|$t$|$) \\ 
    \hline
(Intercept) & -29.901   & 7.789         & -3.839    & 0.001 \\ 
poverty\%   & 2.559     & 0.390         & 6.562     & 0.000 \\ 
    \hline
\end{tabular}
\[ s = 5.512 \qquad R^2 = 70.52\% \qquad R^2_{adj} = 68.89\% \]
\end{center}
\begin{parts}
\item What are the hypotheses for evaluating whether poverty percentage 
is a significant predictor of murder rate?
\item State the conclusion of the hypothesis test from part (a) in 
context of the data.
\item Calculate a 95\% confidence interval for the slope of poverty 
percentage, and interpret it in context of the data.
\item Do your results from the hypothesis test and the confidence 
interval agree? Explain.
\end{parts}
}{}

% 8

\eoce{\qt{Babies\label{babies_head_gestation_inf}} Is the gestational age 
(time between conception and birth) of a low birth-weight baby useful 
in predicting head circumference at birth? Twenty-five low birth-weight 
babies were studied at a Harvard teaching hospital; the investigators 
calculated the regression of head circumference (measured in centimeters) 
against gestational age (measured in weeks). The estimated regression 
line is
\[ \widehat{head~circumference} = 3.91 + 0.78 \times gestational~age \]
\begin{parts}
\item What is the predicted head circumference for a baby whose 
gestational age is 28 weeks?
\item The standard error for the coefficient of gestational age is 0.
35, which is associated with $df=23$. Does the model provide strong 
evidence that gestational age is significantly associated with head 
circumference?
\end{parts} 
}{}

% 9

\eoce{\noindent \begin{minipage}[c]{0.43\textwidth}
\qt{Match the correlation, Part III\label{match_corr_3}} 
\textbf{\color{red}FINAL FORMATTING REQUIRED.}
\textbf{\Large\color{red}MODIFY THIS EXERCISE TO HAVE NEW GRAPHS.}
Match the calculated correlations to the corresponding scatterplot.
\begin{parts}
\item $r = 0.49$
\item $r = -0.48$ 
\item $r = -0.03$ 
\item $r = -0.85$
\end{parts} \vspace{21mm}
\end{minipage}
\begin{minipage}[c]{0.57\textwidth}
\begin{center}
\includegraphics[width=0.45\textwidth]{ch_regr_simple_linear/figures/eoce/match_corr_3/match_corr_1_strong_neg_curved.pdf}
\includegraphics[width= 0.45\textwidth]{ch_regr_simple_linear/figures/eoce/match_corr_3/match_corr_2_weak_pos.pdf} \\
\includegraphics[width= 0.45\textwidth]{ch_regr_simple_linear/figures/eoce/match_corr_3/match_corr_3_n.pdf}
\includegraphics[width= 0.45\textwidth]{ch_regr_simple_linear/figures/eoce/match_corr_3/match_corr_4_weak_neg.pdf}
\end{center}
\end{minipage}
}{}

% 10

\eoce{\qt{True / False\label{tf_correlation}} Determine if the following 
statements are true or false. If false, explain why.
\begin{parts}
\item A correlation coefficient of -0.90 indicates a stronger 
linear relationship than a correlation coefficient of 0.5.
\item Correlation is a measure of the association between any 
two variables.
\end{parts}
}{}

% 11

\eoce{\qt{Trees\label{trees_volume_height_diameter}} The scatterplots below 
show the relationship between height, diameter, and volume of timber 
in 31 felled black cherry trees. The diameter of the tree is measured 
4.5 feet above the ground.\footfullcite{data:trees}
\begin{center}
\includegraphics[width=0.485\textwidth]{ch_regr_simple_linear/figures/eoce/trees_volume_height_diameter/trees_volume_height.pdf}
\hspace{0.02\textwidth}%
\includegraphics[width=0.485\textwidth]{ch_regr_simple_linear/figures/eoce/trees_volume_height_diameter/trees_volume_diameter.pdf}
\end{center}
\begin{parts}
\item Describe the relationship between volume and height of these trees.
\item Describe the relationship between volume and diameter of these trees.
\item Suppose you have height and diameter measurements for another black 
cherry tree. Which of these variables would be preferable to use to predict 
the volume of timber in this tree using a simple linear regression model? 
Explain your reasoning.
\end{parts}
}{}

% 12

\eoce{\qt{Murders and poverty, Part III\label{murders_poverty_inf_one_sided}} 
In Exercises~\ref{murders_poverty_inf} you evaluated whether poverty 
percentage is a significant predictor of murder rate. How, if at 
all, would your answer change if we wanted to find out whether 
poverty percentage is positively associated with murder rate. Make 
sure to include the appropriate p-value for this hypothesis test 
in your answer.
}{}

% 13

\eoce{\qt{Cats, Part II\label{cat_body_heart_inf}} Exercise~\ref{cat_body_heart_reg} 
presents regression output from a model for predicting the heart 
weight (in g) of cats from their body weight (in kg). The coefficients 
are estimated using a dataset of 144 domestic cat. The model output 
is also provided below.
\begin{center}
\begin{tabular}{rrrrr}
    \hline
            & Estimate  & Std. Error    & t value   & Pr($>$$|$t$|$) \\ 
    \hline
(Intercept) & -0.357    & 0.692         & -0.515    & 0.607 \\ 
body wt     & 4.034     & 0.250         & 16.119    & 0.000 \\ 
    \hline
\end{tabular}
\[ s = 1.452 \qquad R^2 = 64.66\% \qquad R^2_{adj} = 64.41\% \]
\end{center}
\begin{parts}
\item What are the hypotheses for evaluating whether body weight is 
positively associated with heart weight in cats? 
\item State the conclusion of the hypothesis test from part (a) in 
context of the data.
\item Calculate a 95\% confidence interval for the slope of body 
weight, and interpret it in context of the data.
\item Do your results from the hypothesis test and the confidence 
interval agree? Explain.
\end{parts}
}{}

% 14

\eoce{$\:$ \\
\noindent \begin{minipage}[c]{0.56\textwidth}
\qt{Nutrition at Starbucks, Part II\label{starbucks_cals_protein}} 
Exercise~\ref{starbucks_cals_carbos} introduced a data set on nutrition 
information on Starbucks food menu items. Based on the scatterplot 
and the residual plot provided, describe the relationship between the 
protein content and calories of these menu items, and determine if a 
simple linear model is appropriate to predict amount of protein from 
the number of calories.
\vspace{34mm}
\end{minipage}
\begin{minipage}[c]{0.02\textwidth}
$\:$ \\
\end{minipage}
\begin{minipage}[c]{0.4\textwidth}
\begin{center}
\includegraphics[width=\textwidth]{ch_regr_simple_linear/figures/eoce/starbucks_cals_protein/starbucks_cals_protein.pdf} \\
\end{center}
\end{minipage}
}{}

% 15

\eoce{\qt{Helmets and lunches\label{helmet_lunch}} The scatterplot shows the 
relationship between socioeconomic status measured as the percentage of 
children in a neighborhood receiving reduced-fee lunches at school 
({\tt lunch}) and the percentage of bike riders in the neighborhood 
wearing helmets ({\tt helmet}). The average percentage of children 
receiving reduced-fee lunches is 30.8\% with a standard deviation 
of 26.7\% and the average percentage of bike riders wearing helmets 
is 38.8\% with a standard deviation of 16.9\%.

\noindent\begin{minipage}[c]{0.56\textwidth}
\begin{parts}
\item If the $R^2$ for the least-squares regression line for these 
data is $72\%$, what is the correlation between {\tt lunch} 
and {\tt helmet}?
\item Calculate the slope and intercept for the least-squares regression 
line for these data.
\item Interpret the intercept of the least-squares regression line in 
the context of the application.
\item Interpret the slope of the least-squares regression line in the 
context of the application.
\item What would the value of the residual be for a neighborhood where 
40\% of the children receive reduced-fee lunches and 40\% of the bike 
riders wear helmets? Interpret the meaning of this residual in the context 
of the application.
\end{parts}
\end{minipage}
\begin{minipage}[c]{0.02\textwidth}
$\:$ \\
\end{minipage}
\begin{minipage}[c]{0.39\textwidth}
\begin{center}
\includegraphics[width=\textwidth]{ch_regr_simple_linear/figures/eoce/helmet_lunch/helmet_lunch.pdf} \\
\end{center}
\end{minipage}
}{}
