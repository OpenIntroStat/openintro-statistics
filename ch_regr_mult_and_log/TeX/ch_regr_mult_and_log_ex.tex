\section{Exercises}

%________________
\subsection{Introduction to multiple regression}

% 1

\eoce{\qt{Baby weights, Part I\label{baby_weights_smoke}} The Child Health 
and Development Studies investigate a range of topics. One study 
considered all pregnancies between 1960 and 1967 among women in the 
Kaiser Foundation Health Plan in the San Francisco East Bay area. Here, 
we study the relationship between smoking and weight of the baby. The 
variable \texttt{smoke} is coded 1 if the mother is a smoker, and 0 if 
not. The summary table below shows the results of a linear regression 
model for predicting the average birth weight of babies, measured in 
ounces, based on the smoking status of the mother. \footfullcite{data:babies}
\begin{center}
\begin{tabular}{rrrrr}
  \hline
            & Estimate  & Std. Error  & t value   & Pr($>$$|$t$|$) \\ 
  \hline
(Intercept) & 123.05    & 0.65        & 189.60    & 0.0000 \\ 
smoke       & -8.94     & 1.03        & -8.65     & 0.0000 \\ 
  \hline
\end{tabular}
\end{center}
The variability within the smokers and non-smokers are about equal and the 
distributions are symmetric. With these conditions satisfied, it is reasonable 
to apply the model. (Note that we don't need to check linearity since the 
predictor has only two levels.)
\begin{parts}
\item Write the equation of the regression line.
\item Interpret the slope in this context, and calculate the predicted birth 
weight of babies born to smoker and non-smoker mothers.
\item Is there a statistically significant relationship between the average 
birth weight and smoking?
\end{parts}
}{}

% 2

\eoce{\qt{Baby weights, Part II\label{baby_weights_parity}} 
Exercise~\ref{baby_weights_smoke} introduces a data set on birth weight of 
babies. Another variable we consider is \texttt{parity}, which is 0 if the 
child is the first born, and 1 otherwise. The summary table below shows the 
results of a linear regression model for predicting the average birth weight 
of babies, measured in ounces, from \texttt{parity}. 
\begin{center}
\begin{tabular}{rrrrr}
  \hline
            & Estimate  & Std. Error  & t value   & Pr($>$$|$t$|$) \\ 
  \hline
(Intercept) & 120.07    & 0.60        & 199.94    & 0.0000 \\ 
parity      & -1.93     & 1.19        & -1.62     & 0.1052 \\ 
  \hline
\end{tabular}
\end{center}
\begin{parts}
\item Write the equation of the regression line.
\item Interpret the slope in this context, and calculate the predicted birth 
weight of first borns and others.
\item Is there a statistically significant relationship between the average 
birth weight and parity?
\end{parts}
}{}

% 3

\eoce{\qt{Baby weights, Part III\label{baby_weights_mlr}} We considered the 
variables \texttt{smoke} and \texttt{parity}, one at a time, in modeling birth 
weights of babies in Exercises~\ref{baby_weights_smoke}
and~\ref{baby_weights_parity}. A more realistic approach to modeling infant 
weights is to consider all possibly related variables at once. Other variables 
of interest include length of pregnancy in days (\texttt{gestation}), mother's 
age in years (\texttt{age}), mother's height in inches (\texttt{height}), and 
mother's pregnancy weight in pounds (\texttt{weight}). Below are three 
observations from this data set. 
\begin{center}
\begin{tabular}{r c c c c c c c}
  \hline
      & bwt & gestation & parity  & age   & height  & weight  & smoke \\ 
  \hline
1     & 120 & 284       & 0       & 27    &  62     & 100     &   0 \\ 
2     & 113 & 282       & 0       & 33    &  64     & 135     &   0 \\ 
$\vdots$ & $\vdots$ & $\vdots$ & $\vdots$ &  $\vdots$ & $\vdots$ & $\vdots$ &   $\vdots$ \\ 
1236  & 117 & 297       & 0       & 38    &  65     & 129     &   0 \\ 
   \hline
\end{tabular}
\end{center}
The summary table below shows the results of a regression model for predicting 
the average birth weight of babies based on all of the variables included in 
the data set.
\begin{center}
\begin{tabular}{rrrrr}
  \hline
            & Estimate  & Std. Error  & t value   & Pr($>$$|$t$|$) \\ 
  \hline
(Intercept) & -80.41    & 14.35       & -5.60     & 0.0000 \\ 
gestation   & 0.44      & 0.03        & 15.26     & 0.0000 \\ 
parity      & -3.33     & 1.13        & -2.95     & 0.0033 \\ 
age         & -0.01     & 0.09        & -0.10     & 0.9170 \\ 
height      & 1.15      & 0.21        & 5.63      & 0.0000 \\ 
weight      & 0.05      & 0.03        & 1.99      & 0.0471 \\ 
smoke       & -8.40     & 0.95        & -8.81     & 0.0000 \\ 
  \hline
\end{tabular}
\end{center}
\begin{parts}
\item Write the equation of the regression line that includes all of the 
variables.
\item Interpret the slopes of \texttt{gestation} and \texttt{age} in this 
context.
\item The coefficient for \texttt{parity} is different than in the linear 
model shown in Exercise~\ref{baby_weights_parity}. Why might there be a 
difference?
\item Calculate the residual for the first observation in the data set.
\item The variance of the residuals is 249.28, and the variance of the birth 
weights of all babies in the data set is 332.57. Calculate the $R^2$ and the 
adjusted $R^2$. Note that there are 1,236 observations in the data set.
\end{parts}
}{}

% 4

\eoce{\qt{Absenteeism, Part I\label{absent_from_school_mlr}} Researchers 
interested in the relationship between absenteeism from school and certain 
demographic characteristics of children collected data from 146 randomly 
sampled students in rural New South Wales, Australia, in a particular school 
year. Below are three observations from this data set. 
\begin{center}
\begin{tabular}{r c c c c}
  \hline
    & eth   & sex   & lrn   & days \\   
  \hline
1   & 0     & 1     & 1     &   2 \\ 
2   & 0     & 1     & 1     &  11 \\ 
$\vdots$ & $\vdots$ & $\vdots$ & $\vdots$ & $\vdots$ \\ 
146 & 1     & 0     & 0     &  37 \\ 
  \hline
\end{tabular}
\end{center}
The summary table below shows the results of a linear regression model for 
predicting the average number of days absent based on ethnic background 
(\texttt{eth}: 0 - aboriginal, 1 - not aboriginal), sex (\texttt{sex}: 0 - 
female, 1 - male), and learner status (\texttt{lrn}: 0 - average learner, 1 - 
slow learner). \footfullcite{data:quine}
\begin{center}
\begin{tabular}{rrrrr}
  \hline
            & Estimate  & Std. Error  & t value   & Pr($>$$|$t$|$) \\ 
  \hline
(Intercept) & 18.93     & 2.57        & 7.37      & 0.0000 \\ 
eth         & -9.11     & 2.60        & -3.51     & 0.0000 \\ 
sex         & 3.10      & 2.64        & 1.18      & 0.2411 \\ 
lrn         & 2.15      & 2.65        & 0.81      & 0.4177 \\ 
  \hline
\end{tabular}
\end{center}
\begin{parts}
\item Write the equation of the regression line.
\item Interpret each one of the slopes in this context.
\item Calculate the residual for the first observation in the data set: a 
student who is aboriginal, male, a slow learner, and missed 2 days of school.
\item The variance of the residuals is 240.57, and the variance of the number 
of absent days for all students in the data set is 264.17. Calculate the $R^2$ 
and the adjusted $R^2$. Note that there are 146 observations in the data set.
\end{parts}
}{}

% 5

\eoce{\qt{GPA\label{gpa}} A survey of 55 Duke University students asked about 
their GPA, number of hours they study at night, number of nights they go out, 
and their gender. Summary output of the regression model is shown below. Note 
that male is coded as 1. 
\begin{center}
\begin{tabular}{rrrrr}
  \hline
            & Estimate  & Std. Error  & t value   & Pr($>$$|$t$|$) \\ 
  \hline
(Intercept) & 3.45      & 0.35        & 9.85      & 0.00 \\ 
studyweek   & 0.00      & 0.00        & 0.27      & 0.79 \\ 
sleepnight  & 0.01      & 0.05        & 0.11      & 0.91 \\ 
outnight    & 0.05      & 0.05        & 1.01      & 0.32 \\ 
gender      & -0.08     & 0.12        & -0.68     & 0.50 \\ 
  \hline
\end{tabular}
\end{center}
\begin{parts}
\item Calculate a 95\% confidence interval for the coefficient of gender in 
the model, and interpret it in the context of the data.
\item Would you expect a 95\% confidence interval for the slope of the 
remaining variables to include 0? Explain
\end{parts}
}{}

% 6

\eoce{\qt{Cherry trees\label{cherry_trees}} Timber yield is approximately 
equal to the volume of a tree, however, this value is difficult to measure 
without first cutting the tree down. Instead, other variables, such as height 
and diameter, may be used to predict a tree's volume and yield. Researchers 
wanting to understand the relationship between these variables for black 
cherry trees collected data from 31 such trees in the Allegheny National 
Forest, Pennsylvania. Height is measured in feet, diameter in inches (at 54 
inches above ground), and volume in cubic feet.\footfullcite{Hand:1994}
\begin{table}[ht]
\begin{center}
\begin{tabular}{rrrrr}
  \hline
            & Estimate  & Std. Error  & t value   & Pr($>$$|$t$|$) \\ 
  \hline
(Intercept) & -57.99    & 8.64        & -6.71     & 0.00 \\ 
height      & 0.34      & 0.13        & 2.61      & 0.01 \\ 
diameter    & 4.71      & 0.26        & 17.82     & 0.00 \\ 
  \hline
\end{tabular}
\end{center}
\end{table}
\begin{parts}
\item Calculate a 95\% confidence interval for the coefficient of height, and 
interpret it in the context of the data.
\item One tree in this sample is 79 feet tall, has a diameter of 11.3 inches, 
and is 24.2 cubic feet in volume. Determine if the model overestimates or 
underestimates the volume of this tree, and by how much.
\end{parts}
}{}


%________________
\subsection{Model selection}

% 7

\eoce{\qt{Baby weights, Part IV\label{baby_weights_model_select_backward}} 
Exercise~\ref{baby_weights_mlr} considers a model that predicts a newborn's 
weight using several predictors (gestation length, parity, age of mother, 
height of mother, weight of mother, smoking status of mother). The table below 
shows the adjusted R-squared for the full model as well as adjusted R-squared 
values for all models we evaluate in the first step of the backwards 
elimination process. 
\begin{center}
\begin{tabular}{rlr}
  \hline
  & Model               & Adjusted $R^2$ \\ 
  \hline
1 & Full model          & 0.2541 \\ 
2 & No gestation        & 0.1031 \\ 
3 & No parity           & 0.2492 \\ 
4 & No age              & 0.2547 \\ 
5 & No height           & 0.2311 \\ 
6 & No weight           & 0.2536 \\ 
7 & No smoking status   & 0.2072 \\ 
  \hline
\end{tabular}
\end{center}
Which, if any, variable should be removed from the model first?
}{}

% 8

\eoce{\qt{Absenteeism, Part II\label{absent_from_school_model_select_backward}} 
Exercise~\ref{absent_from_school_mlr} considers a model that predicts the 
number of days absent using three predictors: ethnic background (\var{eth}), 
gender (\var{sex}), and learner status (\var{lrn}). The table below shows the 
adjusted R-squared for the model as well as adjusted R-squared values for all 
models we evaluate in the first step of the backwards elimination process. 
\begin{center}
\begin{tabular}{rlr}
  \hline
  & Model               & Adjusted $R^2$ \\ 
  \hline
1 & Full model          & 0.0701 \\ 
2 & No ethnicity        & -0.0033 \\ 
3 & No sex              & 0.0676 \\ 
4 & No learner status   & 0.0723 \\ 
  \hline
\end{tabular}
\end{center}
Which, if any, variable should be removed from the model first?
}{}

% 9

\eoce{\qt{Baby weights, Part V\label{baby_weights_model_select_forward}} 
Exercise~\ref{baby_weights_mlr} provides regression output for the full model
(including all explanatory variables available in the data set) for predicting 
birth weight of babies. In this exercise we consider a forward-selection 
algorithm and add variables to the model one-at-a-time. The table below shows 
the p-value and adjusted $R^2$ of each model where we include only the 
corresponding predictor. Based on this table, which variable should be added 
to the model first?\vspace{0.5mm}
\begin{center}
\begin{tabular}{l c c c c c c}
\hline
variable    & gestation             & parity  & age     
                & height          
                    & weight              
                        & smoke \\
\hline
p-value     & $2.2 \times 10^{-16}$ & 0.1052  & 0.2375  
                & $2.97 \times 10^{-12}$
                    & $8.2 \times 10^{-8}$
                        & $2.2 \times 10^{-16}$ \\
$R_{adj}^2$ & 0.1657                & 0.0013  & 0.0003  
                & 0.0386        
                    & 0.0229        
                        & 0.0569 \\
\hline
\end{tabular}
\end{center}
}{}


% 10

\eoce{\qt{Absenteeism, Part III\label{absent_from_school_model_select_forward}} 
Exercise~\ref{absent_from_school_mlr} provides regression output for the full 
model, including all explanatory variables available in the data set, for 
predicting the number of days absent from school. In this exercise we consider 
a forward-selection algorithm and add variables to the model one-at-a-time. 
The table below shows the p-value and adjusted $R^2$ of each model where we 
include only the corresponding predictor. Based on this table, which variable 
should be added to the model first?\vspace{0.5mm}
\begin{center}
\begin{tabular}{l c c c}
  \hline
variable    & ethnicity  & sex     & learner status  \\
  \hline
p-value   & 0.0007     & 0.3142  & 0.5870  \\
$R_{adj}^2$ & 0.0714     & 0.0001  & 0 \\
  \hline
\end{tabular}
\end{center}
}{}

% 11
\eoce{\qt{Movie lovers, Part I\label{movie_lovers_pval_select}} Suppose a 
social scientist is interested in studying what makes
audiences love or hate a movie. She collects a random sample of movies (genre, 
length, cast, director, budget, etc.) as well as a measure of the success of 
the movie (score on a film review aggregator website). If as part of her 
research she is interested in finding out which variables are significant 
predictors of movie success, what type of model selection
method should she use?
}{}

% 12
\eoce{\qt{Movie lovers, Part II\label{movie_lovers_adjrsq_select}} Suppose an 
online media streaming company is interested in
building a movie recommendation system. The website maintains data on the
movies in their database (genre, length, cast, director, budget, etc.) and 
additionally collects data from their subscribers (demographic information, 
previously watched movies, how they rated previously watched movies, etc.). 
The recommendation system will be deemed successful if subscribers actually 
watch, and rate highly, the movies recommended to them. Should the company use 
the adjusted $R^2$ or the p-value approach in selecting
variables for their recommendation system?
}{}


%________________
\subsection{Checking model assumptions using graphs}


% 13

\eoce{\qt{Baby weights, Part V\label{baby_weights_conds}} 
Exercise~\ref{baby_weights_mlr} presents a regression model for predicting the 
average birth weight of babies based on length of gestation, parity, height, 
weight, and smoking status of the mother. Determine if the model assumptions 
are met using the plots below. If not, describe how to proceed with the 
analysis.
\begin{center}
\includegraphics[width=0.4\textwidth]{ch_regr_mult_and_log/figures/eoce/baby_weights_conds/baby_weights_conds_normal_qq.pdf} \hspace{5mm}
\includegraphics[width=0.4\textwidth]{ch_regr_mult_and_log/figures/eoce/baby_weights_conds/baby_weights_conds_abs_res_fitted.pdf}\\[3mm]
\includegraphics[width=0.4\textwidth]{ch_regr_mult_and_log/figures/eoce/baby_weights_conds/baby_weights_conds_res_order.pdf} \hspace{5mm}
\includegraphics[width=0.4\textwidth]{ch_regr_mult_and_log/figures/eoce/baby_weights_conds/baby_weights_conds_res_gestation.pdf}\\[3mm]  
\includegraphics[width=0.4\textwidth]{ch_regr_mult_and_log/figures/eoce/baby_weights_conds/baby_weights_conds_res_parity.pdf} \hspace{5mm}
\includegraphics[width=0.4\textwidth]{ch_regr_mult_and_log/figures/eoce/baby_weights_conds/baby_weights_conds_res_height.pdf}\\[3mm]
\includegraphics[width=0.4\textwidth]{ch_regr_mult_and_log/figures/eoce/baby_weights_conds/baby_weights_conds_res_weight.pdf} \hspace{5mm}
\includegraphics[width=0.4\textwidth]{ch_regr_mult_and_log/figures/eoce/baby_weights_conds/baby_weights_conds_res_smoke.pdf} 
\end{center}
}{}

% 14

\eoce{\qt{GPA and IQ\label{gpa_iq_gender_cond}} A regression model for 
predicting GPA from gender and IQ was fit, and both predictors were found to 
be statistically significant. Using the plots given below, determine if this 
regression model is appropriate for these data.
\begin{center}
\includegraphics[width=0.47\textwidth]{ch_regr_mult_and_log/figures/eoce/gpa_iq_conds/gpa_iq_conds_normal_qq.pdf}\hspace{3mm}
\includegraphics[width=0.47\textwidth]{ch_regr_mult_and_log/figures/eoce/gpa_iq_conds/gpa_iq_conds_abs_res_fitted.pdf}\\[5mm]
\includegraphics[width=0.47\textwidth]{ch_regr_mult_and_log/figures/eoce/gpa_iq_conds/gpa_iq_conds_res_order.pdf}\hspace{3mm}
\includegraphics[width=0.47\textwidth]{ch_regr_mult_and_log/figures/eoce/gpa_iq_conds/gpa_iq_conds_res_iq.pdf}\\[5mm]
\includegraphics[width=0.5\textwidth]{ch_regr_mult_and_log/figures/eoce/gpa_iq_conds/gpa_iq_conds_res_gender.pdf} 
\end{center}
}{}

%________________
\subsection{Introduction to logistic regression}

% 15

\eoce{\qt{Possum classification, Part I\label{possum_classification_model_select}} 
The common brushtail possum of the Australia region is a bit cuter than its 
distant cousin, the American opossum (see Figure~\vref{brushtail_possum}). We 
consider 104 brushtail possums from two regions in Australia, where the 
possums may be considered a random sample from the population. The first 
region is Victoria, which is in the eastern half of Australia and traverses 
the southern coast. The second region consists of New South Wales and 
Queensland, which make up eastern and northeastern Australia.

We use logistic regression to differentiate between possums in these two 
regions. The outcome variable, called \var{population}, takes value 1 when a 
possum is from Victoria and 0 when it is from New South Wales or Queensland. 
We consider five predictors: \var{sex\_\hspace{0.3mm}male} (an indicator for a 
possum being male), \var{head\_\hspace{0.3mm}length}, 
\var{skull\_\hspace{0.3mm}width}, \var{total\_\hspace{0.3mm}length}, and 
\var{tail\_\hspace{0.3mm}length}. Each variable is summarized in a histogram. 
The full logistic regression model and a reduced model after variable 
selection are summarized in the table.

\begin{center}
\includegraphics[width=\textwidth]{ch_regr_mult_and_log/figures/eoce/possum_classification_model_select/possum_variables.pdf} 
\end{center}

\begin{center}\footnotesize
\begin{tabular}{r rrrr r rrrr}
                            & \multicolumn{4}{c}{\emph{Full Model}} &
                            & \multicolumn{4}{c}{\emph{Reduced Model}}  \\
  \cline{2-5}\cline{7-10}
\vspace{-3.1mm} \\
                            & Estimate & SE & Z & Pr($>$$|$Z$|$) &
                            & Estimate & SE & Z & Pr($>$$|$Z$|$) \\ 
  \hline
\vspace{-3.1mm} \\
(Intercept)                 & 39.2349 & 11.5368 & 3.40  & 0.0007 &
                            & 33.5095 & 9.9053  & 3.38  & 0.0007 \\ 
sex\_\hspace{0.3mm}male     & -1.2376 & 0.6662  & -1.86 & 0.0632 &
                            & -1.4207 & 0.6457  & -2.20 & 0.0278 \\ 
head\_\hspace{0.3mm}length  & -0.1601 & 0.1386  & -1.16 & 0.2480 \\ 
skull\_\hspace{0.3mm}width  & -0.2012 & 0.1327  & -1.52 & 0.1294 &
                            & -0.2787 & 0.1226  & -2.27 & 0.0231 \\ 
total\_\hspace{0.3mm}length & 0.6488  & 0.1531  & 4.24  & 0.0000 &
                            & 0.5687  & 0.1322  & 4.30  & 0.0000 \\ 
tail\_\hspace{0.3mm}length  & -1.8708 & 0.3741  & -5.00 & 0.0000 &
                            & -1.8057 & 0.3599  & -5.02 & 0.0000 \\ 
  \hline
\end{tabular}
\end{center}

\begin{parts}
\item Examine each of the predictors. Are there any outliers that are likely 
to have a very large influence on the logistic regression model?
\item The summary table for the full model indicates that at least one 
variable should be eliminated when using the p-value approach for variable 
selection: \var{head\_\hspace{0.3mm}length}. The second component of the table 
summarizes the reduced model following variable selection. Explain why the 
remaining estimates change between the two models.
\end{parts}
}{}

% 16

\eoce{\qt{Challenger disaster, Part I\label{challenger_disaster_model_select}} 
On January 28, 1986, a routine launch was anticipated for the Challenger space 
shuttle. Seventy-three seconds into the flight, disaster happened: the shuttle 
broke apart, killing all seven crew members on board. An investigation into 
the cause of the disaster focused on a critical seal called an O-ring, and it 
is believed that damage to these O-rings during a shuttle launch may be 
related to the ambient temperature during the launch. The table below 
summarizes observational data on O-rings for 23 shuttle missions, where the 
mission order is based on the temperature at the time of the launch. 
\emph{Temp} gives the temperature in Fahrenheit, \emph{Damaged} represents the 
number of damaged O-rings, and \emph{Undamaged} represents the number of
O-rings that were not damaged.
\begin{center}
\begin{tabular}{l rrrrr rrrrr rrrrr rrrrr rrr}
\hline
\vspace{-3.1mm} \\
Shuttle Mission   & 1  & 2 & 3 & 4 & 5 & 6 & 7 & 8 & 9 & 10 & 11 & 12 \\
\hline
\vspace{-3.1mm} \\
Temperature       & 53 & 57 & 58 & 63 & 66 & 67 & 67 & 67 & 68 & 69 & 70 & 70  \\
Damaged           & 5  & 1 & 1 & 1 & 0 & 0 & 0 & 0 & 0 & 0 & 1 & 0 \\
Undamaged         & 1  & 5 & 5 & 5 & 6 & 6 & 6 & 6 & 6 & 6 & 5 & 6 \\
\hline
\\ 
\cline{1-12}
\vspace{-3.1mm} \\
Shuttle Mission   & 13 & 14 & 15 & 16 & 17 & 18 & 19 & 20 & 21 & 22 & 23 \\
\cline{1-12}
\vspace{-3.1mm} \\
Temperature       & 70 & 70 & 72 & 73 & 75 & 75 & 76 & 76 & 78 & 79 & 81 \\
Damaged           & 1  & 0 & 0 & 0 & 0 & 1 & 0 & 0 & 0 & 0 & 0 \\
Undamaged         & 5  & 6 & 6 & 6 & 6 & 5 & 6 & 6 & 6 & 6 & 6 \\
\cline{1-12}
\end{tabular}
\end{center}

\begin{parts}
\item Each column of the table above represents a different shuttle mission. 
Examine these data and describe what you observe with respect to the 
relationship between temperatures and damaged O-rings.
\item Failures have been coded as 1 for a damaged O-ring and 0 for an 
undamaged O-ring, and a logistic regression model was fit to these data. A 
summary of this model is given below. Describe the key components of this 
summary table in words.
\begin{center}
\begin{tabular}{rrrrr}
  \hline
            & Estimate & Std. Error & z value   & Pr($>$$|$z$|$) \\ 
  \hline
(Intercept) & 11.6630  & 3.2963     & 3.54      & 0.0004 \\ 
Temperature & -0.2162  & 0.0532     & -4.07     & 0.0000 \\ 
  \hline
\end{tabular}
\end{center}
\item Write out the logistic model using the point estimates of the model 
parameters.
\item Based on the model, do you think concerns regarding O-rings are 
justified? Explain.
\end{parts}
}{}

% 17

\eoce{\qt{Possum classification, Part II\label{possum_classification_predict}} 
A logistic regression model was proposed for classifying common brushtail 
possums into their two regions in 
Exercise~\ref{possum_classification_model_select}. The outcome variable took 
value 1 if the possum was from Victoria and 0 otherwise.
\begin{center}
\begin{tabular}{r rrrr}
  \hline
\vspace{-3.1mm} \\
                            & Estimate  & SE      & Z     & Pr($>$$|$Z$|$) \\ 
  \hline
\vspace{-3.1mm} \\
(Intercept)                 & 33.5095   & 9.9053  & 3.38  & 0.0007 \\ 
sex\_\hspace{0.3mm}male     & -1.4207   & 0.6457  & -2.20 & 0.0278 \\ 
skull\_\hspace{0.3mm}width  & -0.2787   & 0.1226  & -2.27 & 0.0231 \\ 
total\_\hspace{0.3mm}length & 0.5687    & 0.1322  & 4.30  & 0.0000 \\ 
tail\_\hspace{0.3mm}length  & -1.8057   & 0.3599  & -5.02 & 0.0000 \\ 
  \hline
\end{tabular}
\end{center}
\begin{parts}
\item Write out the form of the model. Also identify which of the variables 
are positively associated when controlling for other variables.
\item Suppose we see a brushtail possum at a zoo in the US, and a sign says 
the possum had been captured in the wild in Australia, but it doesn't say 
which part of Australia. However, the sign does indicate that the possum is 
male, its skull is about 63 mm wide, its tail is 37 cm long, and its total 
length is 83 cm. What is the reduced model's computed probability that this 
possum is from Victoria? How confident are you in the model's accuracy of this 
probability calculation?
%logitp <- 33.5095 - 1.4207 - 0.2787*63 + 0.5687*83 - 1.8057*37; exp(logitp)/(1+exp(logitp))
\end{parts}
}{}

% 18

\eoce{\qt{Challenger disaster, Part II\label{challenger_disaster_predict}} 
Exercise~\ref{challenger_disaster_model_select} introduced us to O-rings that 
were identified as a plausible explanation for the breakup of the Challenger 
space shuttle 73 seconds into takeoff in 1986. The investigation found that 
the ambient temperature at the time of the shuttle launch was closely related 
to the damage of O-rings, which are a critical component of the shuttle. See 
this earlier exercise if you would like to browse the original data.
\begin{center}
\includegraphics[width=0.6\textwidth]{ch_regr_mult_and_log/figures/eoce/challenger_disaster_predict/challenger_disaster_damage_temp.pdf} 
\end{center}
\begin{parts}
\item The data provided in the previous exercise are shown in the plot. The 
logistic model fit to these data may be written as
\begin{align*}
\log\left( \frac{\hat{p}}{1 - \hat{p}} \right) = 11.6630 - 0.2162\times Temperature
\end{align*}
where $\hat{p}$ is the model-estimated probability that an O-ring will become 
damaged. Use the model to calculate the probability that an O-ring will become 
damaged at each of the following ambient temperatures: 51, 53, and 55 degrees 
Fahrenheit. The model-estimated probabilities for several additional ambient 
temperatures are provided below, where subscripts indicate the temperature:
\begin{align*}
&\hat{p}_{57} = 0.341
	&& \hat{p}_{59} = 0.251
	&& \hat{p}_{61} = 0.179
	&& \hat{p}_{63} = 0.124 \\
&\hat{p}_{65} = 0.084
	&& \hat{p}_{67} = 0.056
	&& \hat{p}_{69} = 0.037
	&& \hat{p}_{71} = 0.024
\end{align*}
\item Add the model-estimated probabilities from part~(a) on the plot, then 
connect these dots using a smooth curve to represent the model-estimated 
probabilities.
\item Describe any concerns you may have regarding applying logistic 
regression in this application, and note any assumptions that are required to 
accept the model's validity.
\end{parts}
}{}