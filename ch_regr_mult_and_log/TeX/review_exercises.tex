


%_______________
\section{Review exercises}

% 1

\eoce{\qt{Title\label{mult_regr_ex}}
\textbf{\color{red}CREATE THIS QUESTION.}
\begin{parts}
\item Fact 1.
\item Fact 2.
\item Fact 3.
\item Fact 4.
\end{parts}
}{}

% 2

\eoce{\qt{TITLE\label{log_regr_ex}}
\textbf{\color{red}CREATE THIS QUESTION.}
\begin{parts}
\item Fact 1.
\item Fact 2.
\item Fact 3.
\item Fact 4.
\end{parts}
}{}

% 3

\eoce{\qt{Multiple regression facts\label{mult_regr_facts}}
\textbf{\color{red}PROOF READ and ADD SOLUTIONS.}
Determine which of the following statements are
true and false.
For those statements that are false,
explain why it is false.
\begin{parts}
\item
    If variables are collinear, then removing
    one variable will have no influence on the
    point estimate of another variable's coefficient.
\item
    Suppose a numerical variable $x$ has a coefficient of
    $b_1 = 2.5$ in the multiple regression model.
    Suppose also that the first observation has $x_1 = 7.2$
    the second observation has a value of $x_1 = 8.2$,
    and these two observations have the same values
    for all other predictors.
    Then the predicted value of the second observation
    will be 2.5 higher than the prediction of the first
    observation based on the multiple regression model.
\item
    If a regression model's first variable has
    a coefficient of $b_1 = 5.7$, then if we are
    able to modify an observation so that $x_1$
    takes a value of 1 more than it would otherwise,
    the value $y_1$ for this observation would
    increase by 5.7.
\item
    Suppose we fit a multiple regression model
    based on a data set of 472 observations.
    We also notice that the distribution of the
    residuals includes some skew but does not
    include any particularly extreme outliers.
    Because the residuals are not nearly normal,
    we should not use this model and require
    more advanced methods to model these data.
\end{parts}
}{}

% 4

\eoce{\qt{Logistic regression facts\label{log_regr_facts}}
\textbf{\color{red}PROOF READ and ADD SOLUTIONS.}
Determine which of the following statements are
true and false.
For those statements that are false,
explain why it is false.
\begin{parts}
\item
    Suppose we consider the first two observations
    based on a logistic regression model,
    where the first variable in observation~1
    takes a value of $x_1 = 6$ and observation~2
    has $x_1 = 4$.
%    Each observation has all the same values for the
%    other variables used in the model.
    Suppose we realized we made an error for these
    two observations, and the first observation
    was actually $x_1 = 7$ (instead of~6)
    and the second observation actually had
    $x_1 = 5$ (instead of~4).
    Then the predicted probability from the
    logistic regression model would increase
    the same amount for each observation after
    we correct these variables.
\item
    When using a logistic regression model,
    it is impossible for the model to predict
    a probability that is negative or a probability
    that is greater than 1.
\item
    Because logistic regression predicts probabilities
    of outcomes, observations used to build a logistic
    regression model need not be independent.
\end{parts}
}{}
