


%_______________
\section{Exercises}


%_______________
\subsection{Examining numerical data}

% 38

\eoce{\qt{Mammal life spans\label{mammal_life_spans}} Data were collected on life spans (in 
years) and gestation lengths (in days) for 62 mammals. A scatterplot of life span versus 
length of gestation is shown below. \footfullcite{Allison+Cicchetti:1975}

\noindent\begin{minipage}[c]{0.44\textwidth}
\begin{parts}
\item What type of an association is apparent between life span and length of gestation?
\item What type of an association would you expect to see if the axes of the plot were reversed, i.e. if we plotted length of gestation versus life span?
\item Are life span and length of gestation independent? Explain your reasoning.
\end{parts}\vspace{6mm}
\end{minipage}
\begin{minipage}[c]{0.55\textwidth}
\begin{center}
\includegraphics[width = 0.86\textwidth]{ch_intro_to_data/figures/eoce/mammal_life_spans/mammal_life_spans_scatterplot.pdf}
\end{center}
\end{minipage}
}{}

% 39

\eoce{\qt{Associations\label{association_plots}} Indicate which of the plots show a \\[1mm]
\noindent\begin{minipage}[b]{0.35\textwidth}
\begin{parts}
\item positive association
\item negative association
\item no association
\end{parts}
Also determine if the positive and negative associations are linear or nonlinear. Each 
part may refer to more than one plot. \vspace{30mm}
\end{minipage}
\begin{minipage}[b]{0.62\textwidth}
\hfill\includegraphics[width = 0.95\textwidth]{ch_intro_to_data/figures/eoce/association_plots/association_plots.pdf}
\end{minipage}
}{}

% 40

\eoce{\qt{Office productivity\label{office_productivity}} Office productivity is relatively low 
when the employees feel no stress about their work or job security. However, high levels 
of stress can also lead to reduced employee productivity. Sketch a plot to represent the 
relationship between stress and productivity.
}{}

% 41

\eoce{\qt{Reproducing bacteria\label{reproducing_bacteria}} Suppose that there is only 
sufficient space and nutrients to support one million bacterial cells in a petri dish. 
You place a few bacterial cells in this petri dish, allow them to reproduce freely, and 
record the number of bacterial cells in the dish over time. Sketch a plot representing 
the relationship between number of bacterial cells and time.
}{}

% 42

\eoce{\qt{Sleeping in college\label{college_sleeping}} A recent article in a college newspaper 
stated that college students get an average of 5.5 hrs of sleep each night. A student who 
was skeptical about this value decided to conduct a survey by randomly sampling 25 
students. On average, the sampled students slept 6.25 hours per night. Identify which 
value represents the sample mean and which value represents the claimed population mean.
}{}

% 43

\eoce{\qt{Parameters and statistics\label{parameters_stats}} Identify which value represents 
the sample mean and which value represents the claimed population mean.
\begin{parts}
\item American households spent an average of about \$52 in 2007 on Halloween 
merchandise such as costumes, decorations and candy. To see if this number had changed, 
researchers conducted a new survey in 2008 before industry numbers were reported. The 
survey included 1,500 households and found that average Halloween spending was \$58 per 
household.
\item The average GPA of students in 2001 at a private university was 3.37. A survey 
on a sample of 203 students from this university yielded an average GPA of 3.59 in 
Spring semester of 2012.
\end{parts}
}{}

% 44

\eoce{\qt{Make-up exam\label{makeup_exam}} In a class of 25 students, 24 of them took an exam 
in class and 1 student took a make-up exam the following day. The professor graded the 
first batch of 24 exams and found an average score of 74 points with a standard 
deviation of 8.9 points. The student who took the make-up the following day scored 64 
points on the exam.
\begin{parts}
\item Does the new student's score increase or decrease the average score?
\item What is the new average?
\item Does the new student's score increase or decrease the standard deviation of the 
scores?
\end{parts}
}{}

% 45

\eoce{\qt{Days off at a mining plant\label{days_off_mining}} Workers at a particular mining 
site receive an average of 35 days paid vacation, which is lower than the national 
average. The manager of this plant is under pressure from a local union to increase the 
amount of paid time off. However, he does not want to give more days off to the workers 
because that would be costly. Instead he decides he should fire 10 employees in such a 
way as to raise the average number of days off that are reported by his employees. In 
order to achieve this goal, should he fire employees who have the most number of days 
off, least number of days off, or those who have about the average number of days off?
}{}

% 46

\eoce{\qt{Medians and IQRs} For each part, compare distributions (1) and (2) based on their medians and IQRs. You do not need to calculate these statistics; simply state how the medians and IQRs compare. Make sure to explain your reasoning. 
\begin{multicols}{2}
\begin{parts}
\item (1) 3, 5, 6, 7, 9 \\
(2) 3, 5, 6, 7, 20
\item (1) 3, 5, 6, 7, 9 \\
(2) 3, 5, 7, 8, 9
\item (1) 1, 2, 3, 4, 5 \\
(2) 6, 7, 8, 9, 10
\item (1) 0, 10, 50, 60, 100 \\
(2) 0, 100, 500, 600, 1000
\end{parts}
\end{multicols}
}{}

% 47

\eoce{\qt{Means and SDs} For each part, compare distributions (1) and (2) based on their means and standard deviations. You do not need to calculate these statistics; simply state how the means and the standard deviations compare. Make sure to explain your reasoning. \textit{Hint:} It may be useful to sketch dot plots of the distributions.
\begin{multicols}{2}
\begin{parts}
\item (1) 3, 5, 5, 5, 8, 11, 11, 11, 13 \\
(2) 3, 5, 5, 5, 8, 11, 11, 11, 20 \\
\item (1) -20, 0, 0, 0, 15, 25, 30, 30 \\
(2) -40, 0, 0, 0, 15, 25, 30, 30
\item (1) 0, 2, 4, 6, 8, 10 \\
(2) 20, 22, 24, 26, 28, 30
\item (1) 100, 200, 300, 400, 500 \\
(2) 0, 50, 300, 550, 600
\end{parts}
\end{multicols}
}{}

% 48

\eoce{\qt{Stats scores\label{stats_scores_box}} Below are the final exam scores of twenty 
introductory statistics students.
\begin{center}
57, 66, 69, 71, 72, 73, 74, 77, 78, 78, 79, 79, 81, 81, 82, 83, 83, 88, 89, 94
\end{center}
Create a box plot of the distribution of these scores. The five number summary provided below may be useful.
\begin{center}
\renewcommand\arraystretch{1.5}
\begin{tabular}{ccccc}
Min & Q1    & Q2 (Median)   & Q3    & Max \\
\hline
57  & 72.5  & 78.5          & 82.5  & 94 \\
\end{tabular}
\end{center}
}{}

% 49

\eoce{\qt{Infant mortality\label{infant_mortality}} The infant mortality rate is defined as 
the number of infant deaths per 1,000 live births. This rate is often used as an 
indicator of the level of health in a country. The relative frequency histogram below 
shows the distribution of estimated infant death rates for 224 countries for which such 
data were available in 2014. 
\footfullcite{data:ciaFactbook}

\noindent\begin{minipage}[c]{0.43\textwidth}
\begin{parts}
\item Estimate Q1, the median, and Q3 from the histogram.
\item Would you expect the mean of this data set to be smaller or larger than the 
median? Explain your reasoning.
\end{parts} \vfill \
\end{minipage}
\begin{minipage}[c]{0.52\textwidth}
\hfill\includegraphics[width = 0.85\textwidth]{ch_intro_to_data/figures/eoce/infant_mortality_rel_freq/infant_mortality_rel_freq_hist.pdf}
\end{minipage}
}{}

% 50

\eoce{\qt{Mix-and-match} Describe the distribution in the histograms below and match them to the box plots. \\
\begin{center}
\includegraphics[width=\textwidth]{ch_intro_to_data/figures/eoce/hist_box_match/hist_box_match.pdf}
\end{center}
}{}

% 51

\eoce{\qt{Air quality\label{air_quality_durham}} Daily air quality is measured by the air 
quality index (AQI) reported by the Environmental Protection Agency. This index reports 
the pollution level and what associated health effects might be a concern. The index is 
calculated for five major air pollutants regulated by the Clean Air Act and takes values 
from 0 to 300, where a higher value indicates lower air quality. AQI was reported for a 
sample of 91 days in 2011 in Durham, NC. The relative frequency histogram below shows 
the distribution of the AQI values on these days. \footfullcite{data:durhamAQI:2011}
\begin{minipage}[c]{0.55\textwidth}
\begin{parts}
\item Estimate the median AQI value of this sample.
\item Would you expect the mean AQI value of this sample to be higher or lower than the 
median? Explain your reasoning.
\item Estimate Q1, Q3, and IQR for the distribution.
\item Would any of the days in this sample be considered to have an unusually low or 
high AQI? Explain your reasoning.
\end{parts}
\end{minipage}
\begin{minipage}[c]{0.45\textwidth}
\begin{center}
\includegraphics[width = \textwidth]{ch_intro_to_data/figures/eoce/air_quality_durham/air_quality_durham_rel_freq_hist.pdf} 
\end{center}
\end{minipage}
}{}


% 52

\eoce{\qt{Median vs. mean\label{estimate_mean_median_simple}}
Estimate the median for the 400 observations shown in the histogram, and note whether you expect the mean to be higher or lower than the median.
\begin{center}
\includegraphics[width = 0.6\textwidth]{ch_intro_to_data/figures/eoce/estimate_mean_median_simple/estimate_mean_median_simple.pdf} 
\end{center}}

% 53

\eoce{\qt{Histograms vs. box plots\label{hist_vs_box}} Compare the two plots below. What 
characteristics of the distribution are apparent in the histogram and not in the box 
plot? What characteristics are apparent in the box plot but not in the histogram?
\begin{center}
\includegraphics[width = 0.6\textwidth]{ch_intro_to_data/figures/eoce/hist_vs_box/hist_vs_box.pdf}
\end{center}
}{}

% 54

\eoce{\qt{Marathon winners\label{marathon_winners}} The histogram and box plots below show the distribution of finishing times for male and female winners of the New York Marathon between 1970 and 1999.
\begin{center}
\includegraphics[width=0.6\textwidth]{ch_intro_to_data/figures/eoce/marathon_winners/marathon_winners_hist_box.pdf}
\end{center}
\begin{parts}
\item What features of the distribution are apparent in the histogram and not the box plot? What features are apparent in the box plot but not in the histogram?
\item What may be the reason for the bimodal distribution? Explain.
\item Compare the distribution of marathon times for men and women based on the box plot shown below.
\begin{center}
\includegraphics[width=0.6\textwidth]{ch_intro_to_data/figures/eoce/marathon_winners/marathon_winners_gender_box.pdf}
\end{center}
\item The time series plot shown below is another way to look at these data. Describe what is visible in this plot but not in the others.
\end{parts}
\begin{center}
\includegraphics[width=0.75\textwidth]{ch_intro_to_data/figures/eoce/marathon_winners/marathon_winners_time_series.pdf} \\
\end{center}
}{}


% 55

\eoce{\qt{Distributions and appropriate statistics, Part I\label{dist_shape_pets_dist_height}} 
For each of the following, state whether you expect the distribution to be 
symmetric, right skewed, or left skewed. Also specify whether the mean or 
median would best represent a typical observation in the data, and whether 
the variability of observations would be best represented using the 
standard deviation or IQR. Explain your reasoning.
\begin{parts}
\item Number of pets per household. 
\item Distance to work, i.e. number of miles between work and home.
\item Heights of adult males.
\end{parts}
}{}

% 56

\eoce{\qt{Distributions and appropriate statistics, Part II\label{dist_shape_housing_alcohol_salary}} For each of the following, state 
whether you expect the distribution to be symmetric, right skewed, or left skewed. 
Also specify whether the mean or median would best represent a typical observation 
in the data, and whether the variability of observations would be best represented 
using the standard deviation or IQR. Explain your reasoning.
\begin{parts}
\item Housing prices in a country where 25\% of the houses cost below \$350,000, 
50\% of the houses cost below \$450,000, 75\% of the houses cost below \$1,000,000 
and there are a meaningful number of houses that cost more than \$6,000,000.
\item Housing prices in a country where 25\% of the houses cost below \$300,000, 
50\% of the houses cost below \$600,000, 75\% of the houses cost below \$900,000 
and very few houses that cost more than \$1,200,000.
\item Number of alcoholic drinks consumed by college students in a given week. 
Assume that most of these students don't drink since they are under 21 years old, 
and only a few drink excessively.
\item Annual salaries of the employees at a Fortune 500 company where only a few 
high level executives earn much higher salaries than all the other employees.
\end{parts}
}{}

% 57

\eoce{\qt{TV watchers\label{dist_shape_TV_watchers}} Students in an AP Statistics class 
were asked how many hours of television they watch per week (including online 
streaming). This sample yielded an average of 4.71 hours, with a standard 
deviation of 4.18 hours. Is the distribution of number of hours students watch 
television weekly symmetric? If not, what shape would you expect this distribution 
to have? Explain your reasoning.
}{}

% 58

\eoce{\qt{Exam scores\label{dist_shape_exam_scores}} The average on a history exam 
(scored out of 100 points) was 85, with a standard deviation of 15. Is the 
distribution of the scores on this exam symmetric? If not, what shape would 
you expect this distribution to have? Explain your reasoning.
}{}

% 59

\eoce{\qt{Facebook friends\label{dist_shape_fb_friends}} Facebook data indicate that 
50\% of Facebook users have 100 or more friends, and that the average friend 
count of users is 190. What do these findings suggest about the shape of the 
distribution of number of friends of Facebook users? \footfullcite{Backstrom:2011}
}{}

% 60

\eoce{\qt{A new statistic\label{new_stat}} The statistic $\frac{\bar{x}}{median}$ can 
be used as a measure of skewness. Suppose we have a distribution where all 
observations are greater than 0, $x_i > 0$. What is the expected shape of 
the distribution under the following conditions? Explain your reasoning.
\begin{parts}
\item $\frac{\bar{x}}{median} = 1$
\item $\frac{\bar{x}}{median} < 1$
\item $\frac{\bar{x}}{median} > 1$
\end{parts}
}{}


% 61

\eoce{\qt{Income at the coffee shop\label{income_coffee_shop}} The first histogram 
below shows the distribution of the yearly incomes of 40 patrons at a college 
coffee shop. Suppose two new people walk into the coffee shop: one making 
\$225,000 and the other \$250,000. The second histogram shows the new income 
distribution. Summary statistics are also provided. \\
\begin{minipage}[c]{0.57\textwidth}
\includegraphics[width=\textwidth]{ch_intro_to_data/figures/eoce/income_coffee_shop/income_coffee_shop.pdf}
\end{minipage}
\begin{minipage}[c]{0.4\textwidth}
\begin{center}
\begin{tabular}{rrr}
\hline
        & (1)       & (2) \\ 
\hline
n       & 40        & 42 \\ 
Min.    & 60,680    & 60,680 \\ 
1st Qu. & 63,620    & 63,710 \\ 
Median  & 65,240    & 65,350 \\ 
Mean    & 65,090    & 73,300 \\ 
3rd Qu. & 66,160    & 66,540 \\ 
Max.    & 69,890    & 250,000 \\ 
SD      & 2,122     & 37,321 \\ 
\hline
\end{tabular}
\end{center}
\end{minipage}
\begin{parts}
\item Would the mean or the median best represent what we might think of as a 
typical income for the 42 patrons at this coffee shop? What does this say about 
the robustness of the two measures?
\item Would the standard deviation or the IQR best represent the amount of 
variability in the incomes of the 42 patrons at this coffee shop? What does 
this say about the robustness of the two measures?
\end{parts}
}{}

% 62

\eoce{\qt{Midrange\label{midrange}} The \textit{midrange} of a distribution is defined as 
the average of the maximum and the minimum of that distribution. Is this statistic 
robust to outliers and extreme skew? Explain your reasoning
}{}

% 63

\eoce{\qt{Commute times\label{county_commute_times}} The US census collects data on 
time it takes Americans to commute to work, among many other variables. The 
histogram below shows the distribution of average commute times in 3,143 US 
counties in 2010. Also shown below is a spatial intensity map of the same data.
\begin{center}
\includegraphics[width=0.48\textwidth]{ch_intro_to_data/figures/eoce/county_commute_times/county_commute_times_hist.pdf}
\includegraphics[width=0.48\textwidth]{ch_intro_to_data/figures/eoce/county_commute_times/county_commute_times_map.pdf}
\end{center}
\begin{parts}
\item Describe the numerical distribution and comment on whether or not a log 
transformation may be advisable for these data.
\item Describe the spatial distribution of commuting times using the map below.
\end{parts} 
}{}


% 64

\eoce{\qt{Hispanic population\label{county_hispanic_pop}} The US census collects 
data on race and ethnicity of Americans, among many other variables. The 
histogram below shows the distribution of the percentage of the population 
that is Hispanic in 3,143 counties in the US in 2010. Also shown is a 
histogram of logs of these values.
\begin{center}
\includegraphics[width=0.48\textwidth]{ch_intro_to_data/figures/eoce/county_hispanic_pop/county_hispanic_pop_hist.pdf}
\includegraphics[width=0.48\textwidth]{ch_intro_to_data/figures/eoce/county_hispanic_pop/county_hispanic_pop_log_hist.pdf}
\includegraphics[width=0.48\textwidth]{ch_intro_to_data/figures/eoce/county_hispanic_pop/county_hispanic_pop_map.pdf}
\end{center}
\begin{parts}
\item Describe the numerical distribution and comment on why we might want 
to use log-transformed values in analyzing or modeling these data.
\item What features of the distribution of the Hispanic population in US 
counties are apparent in the map but not in the histogram? What features are 
apparent in the histogram but not the map?
\item Is one visualization more appropriate or helpful than the other? Explain 
your reasoning.
\end{parts} 
}{}


%_______________
\subsection{Considering categorical data}

% 65

\eoce{\qt{Antibiotic use in children\label{antibiotic_use_children}} The bar plot 
and the pie chart below show the distribution of pre-existing medical 
conditions of children involved in a study on the optimal duration of 
antibiotic use in treatment of tracheitis, which is an upper respiratory 
infection.
\begin{center}
\includegraphics[width = 0.45\textwidth]{ch_intro_to_data/figures/eoce/antibiotic_use_children/antibiotic_use_children_bar}
\includegraphics[width = 0.45\textwidth]{ch_intro_to_data/figures/eoce/antibiotic_use_children/antibiotic_use_children_pie}
\end{center}
\begin{parts}
\item What features are apparent in the bar plot but not in the pie chart?
\item What features are apparent in the pie chart but not in the bar plot?
\item Which graph would you prefer to use for displaying these categorical data?
\end{parts}
}{}


% 66

\eoce{\qt{Views on immigration\label{immigration}} 910 randomly sampled registered 
voters from Tampa, FL were asked if they thought workers who have illegally 
entered the US should be (i) allowed to keep their jobs and apply for 
US citizenship, (ii) allowed to keep their jobs as temporary guest workers 
but not allowed to apply for US citizenship, or (iii) lose their jobs and 
have to leave the country. The results of the survey by political ideology 
are shown below.\footfullcite{survey:immigFL:2012}
\begin{center}
\begin{tabular}{l l c c c c}
                        &                           & \multicolumn{3}{c}{\textit{Political ideology}} \\
\cline{3-5}
                        &                           & Conservative  & Moderate  & Liberal   & Total \\
\cline{2-6}
                        & (i) Apply for citizenship & 57            & 120       & 101       & 278 \\
                        & (ii) Guest worker         & 121           & 113       & 28        & 262 \\
\raisebox{1.5ex}[0pt]{\emph{Response}} & (iii) Leave the country    & 179       & 126       & 45        & 350 \\ 
                        & (iv) Not sure             & 15            & 4         & 1         & 20\\
\cline{2-6}
                        & Total                     & 372           & 363       & 175       & 910
\end{tabular}
\end{center}
\begin{parts}
\item What percent of these Tampa, FL voters identify themselves as conservatives?
\item What percent of these Tampa, FL voters are in favor of the citizenship option?
\item What percent of these Tampa, FL voters identify themselves as conservatives 
and are in favor of the citizenship option?
\item What percent of these Tampa, FL voters who identify themselves as 
conservatives are also in favor of the citizenship option? What percent of 
moderates share this view? What percent of liberals share this view?
\item Do political ideology and views on immigration appear to be independent? 
Explain your reasoning.
\end{parts}
}{}

% 67

\eoce{\qt{Views on the DREAM Act\label{dream_act_mosaic}} A random sample of registered 
voters from Tampa, FL were asked if they support the DREAM Act, a proposed law which would provide a path to citizenship for people brought illegally to the US as children.
The survey also collected information on the political ideology of the respondents. 
Based on the mosaic plot shown below, do views on the DREAM Act and  
political ideology appear to be independent? Explain your reasoning.
\footfullcite{survey:immigFL:2012}
\begin{center}
\includegraphics[width = 0.8\textwidth]{ch_intro_to_data/figures/eoce/dream_act_mosaic/dream_act_mosaic.pdf}
\end{center}
}{}


% 68

\eoce{\qt{Raise taxes\label{raise_taxes_mosaic}} A random sample of registered 
voters nationally were asked whether they think it's better to raise taxes 
on the rich or raise taxes on the poor. The survey also collected information 
on the political party affiliation of the respondents. Based on the mosaic 
plot shown below, do views on raising taxes and  
political affiliation appear to be independent? Explain your reasoning.
\footfullcite{survey:raiseTaxes:2015}
\begin{center}
\includegraphics[width = 0.67\textwidth]{ch_intro_to_data/figures/eoce/raise_taxes_mosaic/raise_taxes_mosaic.pdf}
\end{center}
}{}


%_______________
\subsection{Case study: gender discrimination}

% 69

\eoce{\qt{Side effects of Avandia\label{randomization_avandia}} Rosiglitazone is the 
active ingredient in the controversial type~2 diabetes medicine Avandia and has 
been linked to an increased risk of serious cardiovascular problems such as 
stroke, heart failure, and death. A common alternative treatment is pioglitazone, 
the active ingredient in a diabetes medicine called Actos. In a nationwide 
retrospective observational study of 227,571 Medicare beneficiaries aged  
65 years or older, it was found that 2,593 of the 67,593 patients using 
rosiglitazone and 5,386 of the 159,978 using pioglitazone had serious 
cardiovascular problems. These data are summarized in the contingency 
table below. \footfullcite{Graham:2010}
\begin{center}
\begin{tabular}{ll  cc c} 
                                &   & \multicolumn{2}{c}{\textit{Cardiovascular problems}} \\
\cline{3-4} 
                                    &               & Yes   & No        & Total \\
\cline{2-5}
\multirow{2}{*}{\textit{Treatment}} & Rosiglitazone & 2,593 & 65,000    & 67,593 \\
                                    & Pioglitazone  & 5,386 & 154,592   & 159,978 \\
\cline{2-5}
                                    & Total         & 7,979 & 219,592   & 227,571
\end{tabular}
\end{center}
\begin{parts}
\item Determine if each of the following statements is true or false. If false, explain why. \textit{Be careful:} The reasoning may be wrong even if the statement's conclusion is correct. In such cases, the statement should be considered false.
\begin{subparts}
\item Since more patients on pioglitazone had cardiovascular problems (5,386 vs. 2,593), we can conclude that the rate of cardiovascular problems for those on a pioglitazone treatment is higher.
\item The data suggest that diabetic patients who are taking rosiglitazone are more likely to have cardiovascular problems since the rate of incidence was (2,593 / 67,593 = 0.038) 3.8\% for patients on this treatment, while it was only (5,386 / 159,978 = 0.034) 3.4\% for patients on pioglitazone.
\item The fact that the rate of incidence is higher for the rosiglitazone group proves that rosiglitazone causes serious cardiovascular problems.
\item Based on the information provided so far, we cannot tell if the difference between the rates of incidences is due to a relationship between the two variables or due to chance.
\end{subparts}
\item What proportion of all patients had cardiovascular problems?
\item If the type of treatment and having cardiovascular problems were independent, about how many patients in the rosiglitazone group would we expect to have had cardiovascular problems?
\item We can investigate the relationship between outcome and treatment in this study using a randomization technique.  While in reality we would carry out the simulations required for randomization using statistical software, suppose we actually simulate using index cards. In order to simulate from the independence model, which states that the outcomes were independent of the treatment, we write whether or not each patient had a cardiovascular problem on cards, shuffled all the cards together, then deal them into two groups of size 67,593 and 159,978. We repeat this simulation 1,000 times and each time record the number of people in the rosiglitazone group who had cardiovascular problems. Use the relative frequency histogram of these counts to answer (i)-(iii).
\end{parts}
\begin{minipage}[c]{0.5\textwidth}
\begin{subparts}
\item What are the claims being tested?
\item Compared to the number calculated in part (b), which would provide more support for the alternative hypothesis,  \textit{more} or \textit{fewer} patients with cardiovascular problems in the rosiglitazone group?
\item What do the simulation results suggest about the relationship between taking rosiglitazone and having cardiovascular problems in diabetic patients?
\end{subparts}
\end{minipage}
\begin{minipage}[c]{0.5\textwidth}
\includegraphics[width = \textwidth]{ch_intro_to_data/figures/eoce/randomization_avandia/randomization_avandia.pdf} \\
\end{minipage}
}{}

% 70

\eoce{\qt{Heart transplants\label{randomization_heart_transplants}} The Stanford 
University Heart Transplant Study was conducted to determine whether an 
experimental heart transplant program increased lifespan. Each patient 
entering the program was designated an official heart transplant candidate, 
meaning that he was gravely ill and would most likely benefit from a new heart. 
Some patients got a transplant and some did not. The variable \texttt{transplant} 
indicates which group the patients were in; patients in the treatment group got a 
transplant and those in the control group did not. Another variable called 
\texttt{survived} was used to indicate whether or not the patient was alive 
at the end of the study. Of the 34 patients in the control group, 30 died. Of the 69 people in the treatment group, 45 died. \footfullcite{Turnbull+Brown+Hu:1974}
\begin{center}
\includegraphics[width= 0.46\textwidth]{ch_intro_to_data/figures/eoce/randomization_heart_transplants/randomization_heart_transplants_mosaic.pdf}
\includegraphics[width= 0.46\textwidth]{ch_intro_to_data/figures/eoce/randomization_heart_transplants/randomization_heart_transplants_box.pdf}
\end{center}
\begin{parts}
\item Based on the mosaic plot, is survival independent of whether or not the 
patient got a transplant? Explain your reasoning.
\item What do the box plots below suggest about the efficacy (effectiveness) of the heart transplant treatment.
\item What proportion of patients in the treatment group and what proportion of 
patients in the control group died?
\item One approach for investigating whether or not the treatment is effective 
is to use a randomization technique.
\begin{subparts}
\item What are the claims being tested?
\item The paragraph below describes the set up for such approach, if we were 
to do it without using statistical software. Fill in the blanks with a number 
or phrase, whichever is appropriate.
\begin{adjustwidth}{2em}{2em}
We write \textit{alive} on \rule{2cm}{0.5pt} cards representing patients who were 
alive at the end of the study, and \textit{dead} on \rule{2cm}{0.5pt} cards 
representing patients who were not. Then, we shuffle these cards and split them 
into two groups: one group of size \rule{2cm}{0.5pt} representing treatment, and 
another group of size \rule{2cm}{0.5pt} representing control. We calculate the 
difference between the proportion of \textit{dead} cards in the treatment and 
control groups (treatment - control) and record this value. We repeat this 100 
times to build a distribution centered at \rule{2cm}{0.5pt}. Lastly, we calculate 
the fraction of simulations where the simulated differences in proportions are 
\rule{2cm}{0.5pt}. If this fraction is low, we conclude that it is unlikely to 
have observed such an outcome by chance and that the null hypothesis should 
be rejected in favor of the alternative.
\end{adjustwidth}
\item What do the simulation results shown below suggest about the effectiveness 
of the transplant program?
\end{subparts}
\end{parts}
\begin{center}
\includegraphics[width= 0.65\textwidth]{ch_intro_to_data/figures/eoce/randomization_heart_transplants/randomization_heart_transplants_rando.pdf}
\end{center}
}{}
